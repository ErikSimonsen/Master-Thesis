\newglossaryentry{benchmark}
{
  name=Benchmark,
  description={Ein Benchmark im Software Performance Testing ist
      eine Metrik oder ein Bezugspunkt, mit dem Softwareprodukte oder -dienstleistungen verglichen werden können, um deren Qualität zu bewerten.
      Beliebte Benchmarks sind beispielsweise CPU-Auslastung, Speicherverbrauch, Durchsatz, Startzeiten, Ausführungszeiten,
      aber auch Fehlerraten und Fehlertoleranz.
    }
}
\newglossaryentry{latenz}
{
  name=Latenz,
  description={
      Die Latenz ist ein Maß dafür, wie schnell ein Server auf Anfragen des Clients reagiert.
      Die Latenz wird normalerweise in Millisekunden (ms) gemessen und wird oft als Antwortzeit bezeichnet.
      Niedrigere Zahlen bedeuten schnellere Antworten. Die Latenz wird clientseitig gemessen, von der
      Zeit, zu der die Anfrage gesendet wird, bis die Antwort eingeht. Netzwerk-Overhead ist in dieser Zahl enthalten.}
}
\newglossaryentry{durchsatz}
{
  name=Durchsatz,
  description={Der Durchsatz gibt an wieviele Anfragen während eines spezifischen Zeit Intervalls von einem Server verarbeitet werden konnen.
      Der Durchsatz wird normalerweise in Anfragen pro Sekunde (requests/sec) angegeben.}
}
\newglossaryentry{perzentile}
{
  name=Perzentile,
  description=
    {In der Statistik ist ein Perzentil (oder ein Zentil) ein Wert, unter den ein bestimmter Prozentsatz von Werten in seiner Häufigkeitsverteilung fällt.
      Wenn die Antwortzeit im 50. Perzentil 100ms beträgt, bedeutet dies, dass 50\% der Anfragen in 100ms oder weniger zurückgegeben wurden.
    }
}
\newglossaryentry{hdrHistogramm}{
  name=HdrHistogramm,
  description={Histogramme die das Aufnehmen und Analysieren von
      ausgewählten Daten über eine konfigurierbare, ganzzahlige Reichweite und eine konfigurierbare
      Genauigkeit innerhalb dieser Reichweite ermöglichen.
    },
  user1={hdrHistogramme}
}
\newglossaryentry{jaxrsg}{
  name=JAX-RS,
  description={
      Die Jakarta RESTful Web Services (JAX-RS) Spezifikation definiert eine Menge an Java APIs, die die Entwicklung von Web-Services gemäß
      dem Representational State Transfer (REST) Architekturstil ermöglichen. \parencite{JakartaEERest}
    }
}
\newglossaryentry{jpag}{
  name=JPA,
  description={
      Die Jakarta Persistence API Spezifikation definiert eine API, die das Zuordnen und Übertragen von Java-Objekten in eine
      relationale Datenbank erleichtern, um sie dort zu persistieren.
      Die dabei verwendete Technik wird als objektrelationale Abbildung bezeichnet. \parencite{MuellerJPA}
    }
}
\newglossaryentry{ormg}{
  name=ORM,
  description={
      Objektrelationale Abbildung in der Informatik ist eine Programmiertechnik zum Konvertieren von Daten zwischen inkompatiblen Typsystemen unter Verwendung objektorientierter
      Programmiersprachen. Dies erzeugt in der Tat eine "virtuelle Objektdatenbank", die innerhalb der Programmiersprache verwendet werden kann.
      Damit ist es möglich Objekte einer objektorientierten Programmiersprache in einer relationalen, nicht objektorientierten Datenbank abzulegen.
    }
}

\newglossaryentry{io}
{
  name=I/O,
  description={
      In der Informatik bezeichnet input/output(I/O) die Kommunikation eines Informationssystems, wie ein Computer, mit der Außenwelt, möglicherweise
      einem Menschen oder einem anderen Informationssystem. Eingaben sind dabei die Signale und Daten, die das System empfängt und
      Ausgaben sind die vom System gesendeten Signale und Daten. Für einen Computer sind auch die auf ihm laufenden Programme Eingaben.
      Der Begriff kann auch als Teil einer Aktion genutzt werden; "I/O auszuführen" bedeutet, eine Eingabe-oder Ausgabeoperation (I/O-Operation) auszuführen.
      Auf Ebene der Computerarchitektur wird allgmein jegliche Übertragung von Informationen (unabhängig von der Richtung) von einer CPU ausgeführt, indem
      Bereiche des Hauptspeichers über individuelle Befehle beschrieben (Schreiboperation) oder gelesen (Leseoperation) werden.
      Daher wird generell jeglicher Transport von Informationen, beispielsweise das Lesen von Daten einer Festplatte, als I/O bezeichnet.
    }
}

\newglossaryentry{servlet-container}{
  name=Servlet-Container,
  description={
      Ein Servlet ist eine Webkomponente, die von einem Container verwaltet wird und dynamische Inhalte generiert.
      Die Servlet-Spezifikation ist Teil der Jakarta-EE-Technologie und Servlet-Container sind damit fester Bestandteil aller Jakarta-EE-Anwendungsserver.
      Servlets sind Java-Klassen, deren kompilierter Bytecode, dynamisch vom Anwendungsserver geladen und ausgeführt wird und somit Anfragen von
      Clients entgegennehmen und beantworten können. \parencite{JakartaServlet}
    }
}
\newglossaryentry{logische CPU}{
  name=Logische CPU,
  description={
      Die Anzahl der logischen Prozessoren beschreibt die Anzahl der verfügbaren Prozessoren, wenn die physischen CPU-Kerne Hyperthreading unterstützen.
      Durch Hyperthreading ist es möglich mehrere Threads (in der Regel auf 2 limitiert) pro CPU-Kern echt parallel auszuführen.
      Dabei beläuft sich die Zahl der logischen Prozessoren in der Regel auf das doppelte der vorhandenen physischen CPU-Kerne.
    },
  user1={logischen CPU}
}
\newglossaryentry{uber-jar}{
  name=Uber-jar,
  description={
      Ein uber-JAR, auch als \textit{fat JAR} bezeichnet, ist eine Java-Archiv das sowohl alle Klassen des Java-Projektes, als auch alle Klassen der
      Abhängigkeiten enthält. Dadurch kann das Programm als einzige .jar-Datei deployed und in einer Java-Laufzeitumgebung ausgeführt werden.
    }
}
\newglossaryentry{callback}{
  name=Callback,
  description={
      Eine Callback-Funktion, auch als Rückruffunktion bezeichnet, ist eine Funktion, die einer anderen Funktion als Parameter übergeben wird und von dieser
      unter gewissen Bedingungen aufgerufen wird.
    }
}
\newglossaryentry{spag}{
  name=SPA,
  description={
      Eine Single-Page-Application (SPA) ist eine Webanwendung, die mit dem Benutzer interagiert, indem sie den aktuellen Inhalt der Seite mit neuen
      Daten ersetzt, anstatt standardmäßig neue Seiten durch den Browser zu laden.
      Das Ziel besteht dabei in schnelleren Übergangen, wodurch das Gefühlt vermittelt wird bei der Website handele es sich um eine native Anwendung.
      In einer SPA findet nie eine Seitenaktualisierung statt, stattdessen wird gesamte erforderliche HTML-, JavaScript- und CSS-Code entweder vom Browser
      mit einem einzigen Seitenladevorgang abgerufen, oder die entsprechenden Ressourcen werden dynamisch geladen und nach Bedarf zur Seite hinzugefügt,
      normalerweise als Reaktion auf Benutzeraktionen.
    }
}
\newglossaryentry{jsong}{
  name=JSON,
  description={
      Bei JavaScript Object Notation (JSON) handelt es sich um ein einfaches, schlankes Datenaustauschformat.
      Es wurde mit der Absicht entwickelt für Menschen einfach zu lesen, und für Maschinen einfach zu parsen und zu generieren zu sein.
      Es basiert dabei auf einer Untermenge der JavaScript Programmiersprache, ist aber als Textformat komplett unabhängig von Programmiersprachen.
      \parencite{JSON}
    }
}
\newacronym{spa}{SPA}{Single Page Application}
\newacronym{http}{HTTP}{Hypertext Transfer Protocol}
\newacronym{hdrhistogram}{HdrHistogramm}{High dynamic range histogram}
\newacronym{cpu}{CPU}{Central Processing Unit}
\newacronym{ram}{RAM}{Random-Access Memory}
\newacronym{jvm}{JVM}{Java Virtual Machine}
\newacronym{aot}{AOT compilation}{Ahead-Of-Time compilation}
\newacronym{jit}{JIT compilation}{Just-In-Time compilation}
\newacronym{json}{JSON}{JavaScript Object Notation}
\newacronym{orm}{ORM}{Object Relational Mapping}
\newacronym{rest}{REST}{Representational State Transfer}
\newacronym{crud}{CRUD}{Create Read Update Delete}
\newacronym{jax-rs}{JAX-RS}{Jakarta RESTful Web Services (früher: Java API for RESTful Web Services)}
\newacronym{jpa}{JPA}{Jakarta Persistence API (früher Java Persistence API)}
\newacronym{api}{API}{Application Programming Interface}