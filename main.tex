\documentclass[12pt,a4paper]{article}

% Seitengestaltung
\usepackage[left=3cm,right=2cm,top=2.5cm,bottom=1.5cm,includeheadfoot]{geometry}

% Anpassung von LaTeX an die deutsche Sprache
\usepackage[english, ngerman]{babel}
 
% Mathematik
\usepackage{amsmath}

% Grafiken und Bilder
\usepackage{graphicx}
\graphicspath{ {resources/} }
% Schriftfarbe
\usepackage{xcolor}

% Indexerstellung
\usepackage{makeidx}

% Blindtext
\usepackage{blindtext}
\usepackage{lipsum}

% Umlaute
\usepackage[utf8]{inputenc}

% Zitate an Sprache angepasst
\usepackage{csquotes}

% Verbesserte Schrift
\usepackage{microtype}

% Verlinkungen
\usepackage{hyperref}

% Bessere Tabellen
\usepackage{array}
%\usepackage[table]{xcolor}

% Zeichensatz
\usepackage[T1]{fontenc}

% Schriftart
\usepackage{lmodern}

% Figure a b
\usepackage{subcaption}

% Code Highlighting
\usepackage{listings}

% Acronyme und Glossar
\usepackage[acronym, toc]{glossaries}

% Literaturverzeichniss
\usepackage[backend=bibtex,style=authoryear]{biblatex}

\addbibresource{main.bib}
\makeglossaries
\newglossaryentry{benchmark}
{
    name=benchmark,
    description={Ein Benchmark im Software Performance Testing ist
            eine Metrik oder ein Bezugspunkt, mit dem Softwareprodukte oder -dienstleistungen verglichen werden können, um deren Qualität zu bewerten.
            Beliebte Benchmarks sind beispielsweise CPU-Auslastung, Speicherverbrauch, Durchsatz, Startzeiten aber auch Fehlerrate und Fehlertoleranz.
        }
}
\newglossaryentry{latenz}
{
    name=latenz,
    description={
            Die Latenz ist ein Maß dafür, wie schnell ein Server auf Anfragen des Clients reagiert.
            Die Latenz wird normalerweise in Millisekunden (ms) gemessen und wird oft als Antwortzeit bezeichnet.
            Niedrigere Zahlen bedeuten schnellere Antworten. Die Latenz wird clientseitig gemessen, von der
            Zeit, zu der die Anfrage gesendet wird, bis die Antwort eingeht. Netzwerk-Overhead ist in dieser Zahl enthalten.}
}
\newglossaryentry{durchsatz}
{
    name=durchsatz,
    description={Der Durchsatz gibt an wieviele Anfragen während eines spezifischen Zeit Intervalls von einem Server verarbeitet werden konnen.
            Der Durchsatz wird normalerweise in Anfragen pro Sekunde (requests/sec) angegeben.}
}

\newglossaryentry{perzentile}
{
    name=perzentile,
    description=
        {In der Statistik ist ein Perzentil (oder ein Zentil) ein Wert, unter den ein bestimmter Prozentsatz von Werten in seiner Häufigkeitsverteilung fällt.
            Wenn die Antwortzeit im 50. Perzentil 100ms beträgt, bedeutet dies, dass 50\% der Anfragen in 100ms oder weniger zurückgegeben wurden.
        }
}

\newglossaryentry{hdrHistogramm}{
    name=hdrHistogramm,
    description={Histogramme die das Aufnehmen und Analysieren von
            ausgewählten Daten über eine konfigurierbare, ganzzahlige Reichweite und eine konfigurierbare
            Genauigkeit innerhalb dieser Reichweite ermöglichen \parencite{HdrHistogramm}
            TODO: Add JavaEE APIS jax rs, orm, json, publisher, subscriber, IO Operationen, SPA, Servlet-Container
        },
    user1={hdrHistogramme}
}
\newacronym{http}{HTTP}{Hypertext Transfer Protocol}
\newacronym{hdrhistogram}{HdrHistogramm}{High dynamic range histogram}
\newacronym{cpu}{CPU}{Central processing unit}
\newacronym{ram}{RAM}{Random-access memory}
\newacronym{spa}{SPA}{Single page application}

% !TEX root = main.tex
\hypersetup{
	pdftitle={Browserbasiertes Video-Konferenzsystem auf Basis von WebRTC},
	pdfsubject={Browserbasiertes Video-Konferenzsystem auf Basis von WebRTC},
	pdfauthor={Erik Simonsen},
	pdfkeywords={WebRTC, Browser, Javascript, RTCPeerConnection, RTC, Videokonferenz},
	pdfstartpage={1},
	plainpages=false,
	hypertexnames=false
}

\input{todo-definition}
\begin{document}

\begin{minipage}{2.1cm}
	\includegraphics[width=2cm]{resources/fh_logo_klein.jpg}
\end{minipage}
\begin{minipage}{10.0cm}
	Ostfalia - Hochschule für angewandte Wissenschaften\\
	Fakultät Informatik\\
	Institut für Software Engineering
\end{minipage}

\vspace{35mm}

\begin{center}
	{\LARGE Master-Arbeit}
	\\[10mm]
\end{center}

\begin{center}
	\LARGE \textbf{Reactive Programming mit Quarkus\\[28mm]}
\end{center}

\begin{table}[h]
	\centering
	\hspace{50mm}\begin{tabular}{lcll}
		eingereicht von &  & Erik Simonsen          & 70455429 \\

		Erstprüfer:     &  & Prof. Dr. B. Müller    &          \\
		Zweitprüfer:    &  & Prof. Dr. H. Grönniger &          \\
	\end{tabular}
\end{table}

\vspace{30mm}

\begin{table}[h]
	\begin{tabular}{lll}
		Wolfenbüttel, den \today \\
	\end{tabular}
\end{table}
\clearpage

\begin{abstract}
\normalsize
\noindent
%ToDo
\end{abstract}

\input{frontmatter/ehrenwort}
\tableofcontents
% Verzeichnis von Abbildungen
\listoffigures
% Verzeichnis von Listings
\lstlistoflistings
% Verzeichnis von Tabellen
\listoftables

\setcounter{page}{1}
\pagenumbering{arabic}
\pagestyle{headings}

\section{Einleitung}
\label{sec:einleitung}
In der folgenden Arbeit sind die ersten Vorkommen von Begriffen und Abkürzungen die möglicherweise nicht geläufig sind, aber
nicht im Fließtext erklärt werden, mit einem \verb|(*)| gekennzeichnet.
Diese Begriffe werden im Glossar und-/oder Acronymverzeichnis kurz erklärt.

\subsection{Problemstellung}
\label{subsec:problemstellung}
In den letzten Jahren sind die Anforderungen an Webanwendungen durch zunehmende Digitalisierung, und entsprechend steigenden Nutzerzahlen,
sowie stark auf Client-Server-Kommunikation basierenden Architekturen, wie Microservices und \Glspl{spag}(*), erheblich gestiegen.

Während geschäftskritische Anwendungen lange Zeit mit ein paar Tausend Anfragen pro Sekunde als hoch frequentiert
galten, müssen solche Webanwendungen heutzutage in der Lage sein eine vielfach höhere Last bewältigen zu können.
Darüber hinaus müssen Sie \verb|skalierbar| sein, also in der Lage sein auf variable Lasten zu reagieren
und sich anzupassen, ohne das es zu Performance-Einbußen kommt.

Das Standardmodell für Java-basierte Webanwendungen ist das \verb|Thread per Request|-Modell.
Dabei wird jede \acrshort{http}(*)-Anfrage an einen Kernel-Thread gebunden, welcher anschließend die Anfrage sequentiell abarbeitet.
Damit eine möglichst hohe Anzahl an Anfragen parallel abgearbeitet werden kann, wird jedem Thread nur ein Teil der verfügbaren
CPU-Rechenzeit eines CPU-Kerns zugewiesen. Sobald die Rechenzeit eines Threads abgelaufen ist, oder er durch einen blockierenden
Funktionsaufruf in einen inaktiven Zustand versetzt wird, wird der nächste Thread bearbeitet: es erfolgt ein \verb|Threadwechsel|.

Obwohl Threadwechsel, im Gegensatz zu Prozesswechseln, sehr kostengünstig sind, sind sie in diesem Modell ab einer kritischen Anzahl von HTTP-Anfragen
pro Sekunde der begrenzende Skalierungsfaktor. Die Threadwechsel zwischen den Kernelthreads des Betriebssystems erfolgen nicht mehr
schnell genug um jede Anfrage bzw. jeden Thread innerhalb einer akzeptablen Zeit zu bearbeiten.
Diese Begrenzung äußert sich letztendlich in nicht ausreichend skalierendem \Gls{durchsatz}(*) der Anwendung.

\subsection{Ziel der Arbeit}
\label{subsec:ziel}
Um höhere Workloads zu bewältigen und Anwendungen skalierbarer zu machen existieren alternative
Anwendungsmodelle, wie das in dieser Arbeit thematisierte \verb|Reactor-Pattern|.
Das Modell nutzt einen Thread pro CPU-Kern und verzichtet somit auf anwendungsbedingte Threadwechsel.
Die Voraussetzung dafür ist allerdings, dass die vorhandenen Threads niemals in einen inaktiven Zustand geraten,
weswegen Anfragen an externe Komponenten wie Datenbanken oder Webservices den Thread nicht blockieren dürfen, und der vom Ergebniss abhängige
Code reaktiv ausgeführt werden muss: man spricht von \verb|reaktiver Programmierung|. Reaktive Programmierung erfordert das die gesamte Programmlogik jeder
Anwendungsschicht asynchron und eventbasiert strukturiert wird.
Nachdem eine Anfrage an eine externe Komponente abgesetzt wurde, beginnt der Thread bereits mit der Abarbeitung der nächsten
HTTP-Anfrage. Sobald ein Resultat verfügbar ist wird es dem Thread mithilfe eines Events mitgeteilt, welcher daraufhin
den, für dieses Event, hinterlegten Code ausführt. Mit dieser grundlegenden Funktionsweise wird auf anwendungsbedingte
Threadwechsel verzichtet.

In dieser Arbeit wird untersucht, ob \verb|Reactive Programming| bzw. reaktive Anwendungen das Problem der Skalierbarkeit
für hohe Lasten praktikabel lösen können und welche alternativen Lösungsansätze es gibt.

\subsection{Vorgehensweise}
\label{subsec:vorgehensweise}
Um das Verhalten der beiden Modelle unter Last zu prüfen und miteinander zu vergleichen, werden zwei triviale Anwendungen,
reaktiv und nicht-reaktiv implementiert, einer Reihe von Lasttests mit unterschiedlichen \verb|workloads| unterzogen.
Dabei werden verschiedene \Glsplural{benchmark}(*), wie Durchsatz, Speicherbedarf und CPU-Auslastung gemessen.
Die Anwendungen werden sowohl als \acrshort{jvm}(*)-Anwendung, als auch als nativ ausführbare Anwendung jeweils mit Datenbankenanbindung und
ohne getestet.

\subsection{Aufbau}
\label{subsec:aufbau}
Zu Beginn der Arbeit werden in Kapitel \ref{sec:grundlagen} die Grundlagen der Thematik erläutert, also Kernel- und User-Threads,
blockierende und nicht-blockierende Eingabe-/Ausgabe-Operationen, sowie die beiden darauf aufbauenden Design-Patterns
\verb|Thread per Request| und \verb|Reactor|.

Anschließend werden in Kapitel \ref{section:reaktive_programmierung} die grundlegenden Konzepte von reaktiver Programmierung,
wie reaktive Datenströme, erklärt, sowie dessen Vor- und Nachteile genannt. Darauffolgend werden die Eigenschaften von reaktiven
Systemen und deren Beziehung untereinander beschrieben.
Danach wird auf die Unterstützung für reaktive Anwendungen und Architekturen innerhalb des Java Ökosystem, sowie Alternativen
eingegangen.

In Kapitel \ref{section:vergleich_reaktiv_blockierend} werden zuerst die Testumgebung definiert und der Testaufbau,
sowie der Testablauf beschrieben.
Im Anschluss werden die Testresultate in Form von Diagrammen dargestellt und beschrieben.
Zuletzt erfolgt die Auswertung und Erläuterung der Testresultate.
\newpage
% !TEX root = main.tex

\section{Grundlagen}
\label{section:grundlagen}

\subsection{Threads \& Prozesse in Java}
\label{sections:treads_prozesse}
In der Java-Laufzeitumgebung sind Prozesse und Threads als Betriebssystem-Prozesse realisiert, sog. \textit{native threads} oder auch \textit{Kernelthreads}.
Hierbei wird die Ausführungsreihenfolge, die Ausführungszeit und der Prozess- \& Threadwechsel
vom Scheduler \& Dispatcher des Betriebssystems übernommen \parencite{Tanenbaum2016}.
Die Threading-Abstraktion in Java bietet Entwicklern verhältnismäßig leichten Zugriff auf parallelle Programmierung und Synchronisation von Threads.\newline
Servlet-Container binden üblicherweise jede vom Webserver weitergeleitete Anfrage an einen
Thread\footnote{Aus einem, im vornherein erzeugten, Thread-Pool} im Servlet-API, welcher die jeweilige Anfrage imperativ abarbeitet
(daher auch \textit{worker thread}).\newline
Der auszuführende Code ist in diesem Ansatz an den jeweiligen Thread gekoppelt, dieser wartet bei
asynchronen Ereignissen solange, bis er eine Antwort erhält und blockiert den Thread bis dahin.

Um jede Anfrage, und somit jeden Thread, scheinbar parallel zu bearbeiten wird vom Scheduler
des Betriebssystems regelmäßig ein Kontextwechsel zwischen den Threads,
ein \textit{thread-switch}, durchgeführt. Während bei einem Prozesswechsel der gesamte Programmkontext (Adressraum, Inhalt der CPU-Register,
Seitentabelle, geöffnete Dateien und Metainformationen)
gewechselt werden muss, wird bei einem Threadwechsel lediglich der Inhalt der CPU-Register (inkl. Programmzähler) ersetzt\parencite{Brosenne2021}\parencite{Mosberger2002}.
Da der Kontextwechsel, im Fall von \textit{native threads}, durch Systemaufrufe, also vom Kernel des Betriebssystems, ausgeführt werden muss, entsteht auch
bei einem Threadwechsel ein messbarer Zeitverlust.\newline
Weitere Threadwechsel entstehen, wenn ein Thread die zugewiesene Rechenzeit nicht nutzen kann, da er noch durch ein asynchrones Ereignis
(abgesetzte Datenbankabfragen oder weitere aufgerufene Webservices) blockiert, und diese einem anderen Thread zugeteilt wird.

Während dieser Zeitverlust für hoch frequentierte Anwendungen lange Zeit kein Problem darstellte,
sind die Anforderungen an Webanwendungen in den letzten Jahren durch steigende Nutzerzahlen und Architekturen,
die stark auf Client-Server-Kommunikation basieren, erheblich gestiegen.
Ab einer gewissen Menge an Anfragen stellen die Kosten der Threadwechsel von \textit{native threads} ein Performance Bottleneck
(in Form von Durchsatz) dar.

\subsection{Reaktive Programmierung}
\label{section:reaktive_programmierung}
Reaktive Programmierung ist ein Programmierparadigma, bei dem der Programmablauf als Sequenz von asynchronen Ereignissen (Events), und
Daten als -von außen- unveränderliche (immutable) Datenströme (Streams) dargestellt werden.
Sobald es innerhalb des Datenstroms zu Änderungen kommt werden diese als Events durch einen Publisher veröffentlicht.\footnote{Änderungen in Datenströmen
    sind der quasi der Stimulus}
Die eigentliche Programmlogik wird in Funktionen ausgeführt, die auf die veröffentlichten Events hören (Subscriber), sie verarbeiten und wiederrum welche
veröffentlichen können. Die Grundidee orientiert sich am Observer-Pattern und dessen Ausprägung: dem Publish-Subscribe Pattern, erweitert diese aber
noch um die Benachrichtigungen des Subscribers:
\begin{enumerate}
    \item Sobald keine Events mehr kommen
    \item Wenn ein Fehler aufgetreten ist
\end{enumerate}
Indem Änderungen direkt propagiert werden und Subscriber keine Kontrolle über den Datenfluss haben, sondern lediglich über Änderungen informiert werden,
können Programme ohne jeglichen Zustand realisiert werden\parencite{Escoffier2017}.


Reaktive Programmierung verinnerlicht das Konzept von nicht-blockierender bzw. asynchroner Ein- und Ausgabe (I/O).
Dabei wird, statt wie bei synchroner bzw. blockierender Ein- und Ausgabe die restliche Ausführung des Programms
zu blockieren bis die Datenübertragung abgeschlossen ist, nach dem Start der Übertragung bereits begonnen die Teile des
Programms auszuführen, die nicht von dem Ergebnis der I/O Operation abhängen.

Dadurch können mehrere parallele Anfragen von dem gleichen Thread bearbeitet werden.
Methoden die blockierende I/O Operationen ausführen, wie Datenbankzugriffe oder Anfragen von externen Services,
geben beim Aufruf unverzüglich einen Publisher zurück, auf dem sich der Aufrufer registriert (subscribe).
Dadurch wird der bearbeitende Thread nicht blockiert, und kann die nächste Anfrage bearbeiten.
Sobald das Ergebnis der I/O Operation bereit ist, wird es dem Publisher in Form eines Events mitgeteilt, von diesem veröffentlicht und die Anfrage kann
vom Aufrufer bzw. Subscriber weiter bearbeitet werden.

Da durch dieses Modell der auszuführende Code nicht mehr an den jeweils ausführenden Thread gebunden wird, erlaubt es die Nutzung eines einzigen
Threads (\textit{sog. IO Thread}) statt eines \textit{Threadpools}.
Dadurch ergeben sich folgende Vorteile:

\begin{enumerate}
    \item Die Antwortenzeiten sind für eine hohe Last geringer, da deutlich weniger Threadwechsel gemacht werden müssen
    \item Der Speicherverbrauch ist geringer, da weniger Threads genutzt werden
    \item Der Grad der Paralellität ist nicht von der Anzahl der Threads begrenzt
\end{enumerate}

Allerdings gibt es auch einige gravierende Nachteile:

\begin{enumerate}
    \item Asynchroner Code ist schwieriger zu schreiben, lesen, testen und zu debuggen als imperativer Code
    \item Sehr aufwendig in bestehende klassische Anwendungen zu integrieren
    \item Reaktive Anwendungen müssen in jeder Schicht reaktiv sein (Transaktionen, Security, Datenbanktreiber)
    \item Da nur ein IO-Thread genutzt wird resultieren blockierende I/O-Operationen in der Blockierung der gesamten Anwendung
\end{enumerate}

Ein beliebtes Threading-Modell für die Verarbeitung von asynchronen Events ist die Event Loop. Sobald ein Event entsteht wird es einer Warteschlange (Queue)
in der Event Loop hinzugefügt. Solange der Main-Thread aktiv ist und die Queue noch Events enthält, wird in einer Schleife das nächste Event
abgerufen und an den, für diesen Eventtyp, registrierten Event-Handler bzw. oben beschriebenen Subscriber weitergeleitet.
Dies können beispielsweise I/O-Events sein, die signalisieren das Daten bereit zur Weiterverarbeitung sind, aber auch jegliches andere Event.
Die Event Loop wird in der Regel vom Main-Thread (in diesem Fall dem genannten IO-Thread) ausgeführt, daher darf das Verarbeiten von Events
keine blockierenden, oder zeitintensiven Operationen beinhalten\parencite{Ponge2020}.

\subsubsection{Alternativen}
In Java 1.1 wurden Threads als sog. \textit{Green Threads} implementiert. Dabei wurde die Möglichkeit Threads vom Betriebssystem verwalten zu lassen 
gar nicht genutzt. Stattdessen lief die komplette JVM in einem einzigen Prozess. 
\textit{Green threads} waren als \textit{user threads} implementiert \footnote{Auch als \textit{Fiber} oder \textit{virtual thread} bezeichnet},
 dabei ist die Funktionalität
nicht im Kernel implementiert (wie bei \textit{kernel-/native threads}), sondern in einer Programmbibliothek im \textit{Userspace}.
Da sich das Betriebssystem nicht um das Scheduling von \textit{user threads} kümmert, wurde dies über einen eigenen Scheduling-Algorithmus der JVM
geregelt.
Ein \textit{green thread} existiert lediglich als Objekt innerhalb der JVM, und durch die virtuelle Speicherverwaltung entfallen somit
 die aufwändigen Betriebssystemaufrufe beim 
Erstellen eines Threads, sowie bei Thread- bzw. Kontextwechseln, denn der ausführende Main-Thread bleibt gleich.
\footnote{Jeder Prozess hat mind. einen Thread (kernel-thread) der das Programm ausführt. Dieser wird daher auch als \textit{main-thread} bezeichnet.}

Die Threadwechsel der \textit{user-threads} erfolgten ausschließlich innerhalb des Main-Threads, weswegen keine echte Parallelität realisiert werden konnte, da immer nur ein 
Prozessorkern genutzt wurde. 
Während der Vorteil dieses Modells darin lag, dass es keine 'echten' parallelen Zugriffe auf eine Resource innerhalb des JVM-Prozesses geben konnte und die Synchronisation
von Datenzugriffen daher leicht war, überwog schließlich der Umstand, dass keine Nutzung von mehreren Prozessorkernen durch Multithreading möglich war
, weswegen \textit{Green Threads} ab Java 1.3 zugunsten von \textit{native threads} entfernt wurden.

Mit dem OpenJDK Projekt \textit{Project Loom} ist die Idee von \textit{Green threads}
 wieder aufgegriffen worden, allerdings nun als Ergänzung (statt Alternative) zu \textit{native threads}. 
Statt alle virtuellen Threads auf dem nativen Main-Thread auszuführen, werden diese von einer geringen Anzahl an nativen \textit{worker threads},
die als Carrier eingesetzt werden, ausgeführt.
Deren Anzahl ist so gewählt, dass alle CPU Kerne durch Multithreading dauerhaft benutzt werden können 
\footnote{In der Praxis laufen natürlich noch andere Prozesse auf dem Server, deren Threads auch ausgeführt werden müssen.}
, aber so wenig Kontextwechsel wie möglich ausgeführt werden müssen. \footnote{Im Idealfall würde auf jedem CPU Kern ein \textit{worker thread} laufen, ohne jemals
einen Thread- bzw. Kontextwechsel machen zu müssen.}
Während ein nativer Thread in einer 64 Bit JVM ca. 1 MB für den Threadstack reserviert und zusätzlich noch Metadaten abspeichert, ist ein virtueller Thread 
lediglich ein Objekt im virtuellen Speicher der JVM und benötigt darüber hinaus für sich selber extrem wenig Resourcen (da er ja im Hintergrund von einem 
nativen Thread abgearbeitet wird). 
Aus diesem Grund können durchaus mehrere Millionen virtueller Threads erzeugt werden (bei entsprechendem allokierten Heap-Speicher der JVM), wohingegen
die Erstellung von 10.000 nativen Threads entweder den allokierten Speicher weit überschreitet (wodurch der JVM-Prozess abstürzt) oder die Threadgrenze
des Betriebssystems wird überschritten.

Sobald ein virtueller Thread nun eine blockierende I/O Operation ausführt signalisiert er dem darunterliegenden nativen Thread, dass er momentan nichts machen kann 
außer Warten, und erlaubt dem nativen Thread zu einem anderen virtuellen Thread zu wechseln.

Das große Versprechen des Projektes ist außerdem, das Entwickler keine asynchronen Programmierparadigmen (wie u.A. \textit{Reactive Programming})
 nutzen müssen um die beschriebenen virtuellen Threads (und die damit einhergehenden wesentlichen Performanceverbesserungen) nutzen zu können.
 Um dieses Versprechen zu halten werden virtuelle Threads, statt als Bibliothek eines Drittanbieters, in Form einer eigenen JDK Version bereitgestellt. 
 In dieser Version wurden viele Teile der Standardbibliothek die mit I/O-Operationen arbeiten so angepasst, dass virtuelle Threads statt native Threads 
 genutzt werden. Auf diese Weise können I/O-Operationen, wie beispielsweise der Aufruf einer Netzwerkfunktionalität, ohne Änderungen am Programm
 die virtuellen Threads nutzen und blockieren den darunterliegenden nativen Thread nicht mehr.

\url{https://jaxenter.de/reactive-programming-82051}
\url{https://jaxenter.de/java/project-loom-besser-skalieren-durch-virtuelle-threads-2-94424}
\url{https://blogs.oracle.com/javamagazine/going-inside-javas-project-loom-and-virtual-threads}
\url{https://inside.java/2021/05/10/networking-io-with-virtual-threads/}

\parencite{Oracle2021}
\subsection{Reaktive Datenströme}
\label{section.reaktive_datenströme}
\url{http://www.reactive-streams.org/}

\subsection{Reaktive Systeme}
\label{section:reaktive_systeme}
\url{https://www.reactivemanifesto.org/}

\subsubsection{Eigenschaften}

\subsubsection{Anwendungsgebiete}

\subsection{Werkzeuge}
\subsubsection{Java Ökosystem}
// CompletableFuture
// Reactive streams
// Flow Api
// JavaRx, Vert.x \& Mutiny in Quarkus
\subsubsection{Andere}


% Definiere Style für Text Listings.
\lstdefinestyle{text}{
    basicstyle=\ttfamily\footnotesize
}
% Verwende Style 
\lstset{style=text}
% !TEX root = main.tex
\section {Vergleich reaktive \& imperative Anwendung}
\label{section:vergleich_reaktiv_imperativ}
Um zu prüfen, ob Leistungsfähigkeit und Skalierbarkeit einer reaktiven Anwendung tatsächlich die einer traditionellen, imperativen Anwendung
übertrifft, werden in diesem Kapitel beide Ansätze hinsichtlich verschiedener Metriken in einem festen Zeitintervall miteinander verglichen.

\subsection{Implementierung \& Systemaufbau}
\label{section:implementierung}
Die beiden Anwendungen implementieren mit dem Quarkus-Framework jeweils eine simple REST-Schnittstelle mit HTTP-CRUD Methoden
und einer angebundenen PostgreSQL-Datenbank.
Dabei ist vorallem die HTTP-Schicht von Interesse. Die HTTP-Unterstützung von Quarkus basiert auf einem reaktiven, nicht-blockierenden
Unterbau: der Vert.x Engine.
Jede HTTP-Anfrage wird auf einem der \textit{event-loop threads} bzw. \textit{IO threads}
\footnote{Deren Anzahl hängt von der Anzahl der CPU-Kerne ab}
verarbeitet und durch eine Routing-Schicht an den Anwendungscode weitergeleitet.
Je nachdem welcher Ansatz zur Implementierung des jeweiligen HTTP-Endpunktes gewählt wurde,
wird der Code dann auf einem blockierenden \textit{worker thread} (Servlet, JAX-RS) oder einem der
\textit{IO threads} (Reactive Routes, Reactive Resteasy) ausgeführt.
Die \textit{IO threads} sind dafür zuständig alle IO-Operationen asynchron auszuführen und die jeweiligen EventListener bzw. Subscriber auszulösen sobald
die Operationen abgeschlossen sind.
\newpage
\begin{figure}[h!]
    \centering
    \includegraphics[width=1.0\textwidth]{Quarkus_HTTP_Layer}
    \caption{Quarkus HTTP-Schicht \parencite{QuarkusReactiveRoutes}}
\end{figure}

Damit sich beide Anwendungen nahe an einer realen Java-EE REST-API orientieren, haben
sie (zusätzlich zu den grundlegenden Abhängigkeiten des Quarkus-Frameworks) folgende Projekt-Abhängigkeiten:
\begin{enumerate}
    \item JAX-RS Implementierung
    \item JSON Unterstützung
    \item Datenbanktreiber
    \item JPA Implementierung
\end{enumerate}

Diese Abhängigkeiten wurden vom Quarkus Maven-Repository sowohl in blockierender,
als auch in nicht-blockierender, reaktiver Form bereitgestellt: \parencite{MavenQuarkusIO}
% space between the text and the left/right border of its containing cell is set to 18 pt
\setlength{\tabcolsep}{18pt}
% the height of each row is set to 1.5 relative to its default height
\renewcommand{\arraystretch}{1.5}
\begin{table}[h!]
    \centering
    \begin{tabular}{| c | c | c |}
        \hline
                         & Blockierend      & Nicht-blockierend (reaktiv) \\
        \hline
        JAX-RS           & Resteasy         & Resteasy Reactive           \\
        \hline
        JSON             & Resteasy-Jackson & Resteasy-Reactive-Jackson   \\
        \hline
        Datenbanktreiber & JDBC-Postgresql  & Reactive-Pg-Client          \\
        \hline
        JPA-/ORM         & Hibernate-ORM    & Hibernate-Reactive          \\
        \hline
    \end{tabular}
    \caption{Tabelle mit den verwendeten Anwendungen beider Applikationen}
    \label{table:dependencies}
\end{table}
Der Projekt-Code dieser Arbeit kann vom Gitlab-Server der Ostfalia unter
\url{//TODO Ostfalia Gitlab link?} eingesehen und abgerufen werden.
\subsection{Testumgebung}
\label{section:testumgebung}
Für die Testumgebung werden zwei Systeme benötigt: der Client-Host und der Server-Host.
Dabei muss es sich um UNIX-Systeme handeln, da eine einige der verwendeten Werkzeuge nur 
auf diesen Systemen verfügbar sind.
Zudem müssen beide Systeme per SSH von einem (idealerweise vorhandenem) dritten System \footnote{Dies kann allerdings auch der Client-Host selber sein} 
erreicht werden können, damit dieses den Testablauf in der korrekten Abfolge ausführen kann. 
Der Einfachheit halber empfiehlt es sich, dass sich alle Geräte im gleichen Netzwerk befinden.
Beiden Anwendungen verwenden Version 2 des Quarkus Frameworks. //TODO: Anwendungen testen nach Umstellung auf V2

Die vom Autor genutzten Client- und Server-Host Systeme zur Durchführung der Tests haben die folgenden Systemspezifikationen:

\begin{table}[h!]
    \centering
    \begin{tabular}{| c | c |}
        \hline
        Server-Host\\
        \hline
        CPU's           & Resteasy         \\
        \hline
        RAM            & Resteasy-Jackson \\
        \hline
        Speicher & JDBC-Post        \\
        \hline
        Betriebssystem         & Hibernate-ORM    \\
        \hline
        Kernel &
        \hline
    \end{tabular}
    \caption{Tabelle mit den verwendeten Anwendungen beider Applikationen}
    \label{table:dependencies}
\end{table}

\begin{table}[h!]
    \centering
    \begin{tabular}{| c | c |}
        \hline
        Client-Host\\
        Hardware & Acer Aspire VN7-591G \\
        \hline
        CPU      & Intel® Core™ i5-4210H CPU @ 2.90GHz × 4          \\
        \hline
        RAM      & 8GB \\
        \hline
        Speicher & 500 GB   \\
        \hline
        Betriebssystem  & Fedora 34 (Workstation Edition)   \\
        \hline
        Kernel & Linux version 5.12.13-300.fc34.x86_64 (mockbuild@bkernel01.iad2.fedoraproject.org) #1 SMP Wed Jun 23 16:18:11 UTC 2021
        \hline
    \end{tabular}
    \caption{Systemspezifikationen der verwendeten Client-Maschine}
    \label{table:dependencies}
\end{table}

\subsection{Vorgehen des Tests / Testaufbau}
\label{section:vorgehen}
//Erwähnen das Testaufbau auf einem von Red Hat durchgeführten Versuch basiert und stark erweitert wurde vom Author
//TODO: Erklären wie das gesamte System getestet wird, welche Werkzeuge ssh, (wrk2 (warum wrk2, latenzmessung im gegensatz zu wrk), top, jbang etc.
docker für reproduzierbare umgebung, lua histogramme dann durch javascript-script zu graphen )
Systemaufbau (Client-Host, Server-Host, User-Host)
Architekturaufbau als Grafik
//eine minute lang http anfragen an zwei Anwendungen, die genau dasselbe machen
was genau gemessen wird (welche Metriken) und wie
diees beeeinflusst werden können und wie diese durch top & wrk ermittelt werden
Jeden größeren Schritt erklären, Bauen der Anwendungen, Warm-Up (JIT), Workload
//TODO: Grafik zur Visualisierung (Illustrator)
\subsection{Test: Statische Daten}
\label{section:statische_daten}

\subsubsection{Systemablauf}
//TODO: Grafik ähnlich zu Grafik in Implementierung aber mit Threadwechseln und exemplarisch mhrere Threads zeigen
(auch angeben wie Ergebnisse mit komplizierteren Queries aussehen könnten)
\subsubsection{Resultate}

\subsection{Test: Datenbankzugriffe}
\label{section:datenbankzugriffe}

\subsubsection{Systemablauf}
//TODO: Grafik ähnlich zu Grafik in Implementierung aber ohne Threadwechsel dafür Main-Thread mit dahinterliegender Datenbank,
das Nicht BLockieren bzw. Asynchronität verdeutlichen

\subsubsection{Resultate}

\subsection{Auswertung}
\label{section:auswertung}
\section{Fazit}
\label{sec:fazit}
\subsection{Zusammenfassung}
\label{subsec:zusammenfassung}
Das Ziel der Masterarbeit war es, zu untersuchen ob reaktive, auf \verb|Nonblocking I/O| und dem \verb|Multi-Reactor|-Modell basierende
Anwendungen das Problem der begrenzten Skalierbarkeit von nicht-reaktiven, auf \verb|Blocking I/O| und dem \verb|Thread per Request|-Modell
basierenden Anwendungen durch Vermeidung von Threadwechseln lösen können, sowie mögliche Alternativen zu beschreiben.

Für diesen Zweck wurden zwei simple Anwendungen als REST-APIs geschrieben, jeweils reaktiv und nicht-reaktiv, deren
leistungsrelevante Metriken durch mehrere Reihen von Lasttests mit variierenden \verb|workloads| gemessen wurden.
Diese Metriken sind die CPU-Auslastung, der Speicherverbrauch, die Startzeit bis zur Bearbeitung der ersten Anfrage,
sowie der Durchsatz und die durchschnittliche Latenz.
Die Anwendungen wurden sowohl im \verb|JVM mode| als auch im \verb|native mode| jeweils mit statischen als auch mit dynamischen
Ressourcen, also einer Datenbankanbindung, getestet.
Anschließend wurden die resultierenden Benchmarks der beiden Anwendungen im jeweiligen Modus für jeden Ressourcentyp ausgewertet und
miteinander verglichen.\newline
Zusammenfassend lassen sich folgende Ergebnisse nennen:\newline
Generell allokieren die \verb|native images| weniger Speicher und benötigen zum Starten nur wenige Millisekunden, was nur ein Bruchteil der Startzeit
von \verb|JVM|-Anwendungen darstellt. Allerdings können sie nur einen deutlich
geringeren maximalen Durchsatz aufweisen, da auf eine Vielzahl an Laufzeitoptimierungen verzichtet wird.
Sowohl für statische, als auch für dynamische Daten beträgt der maximale Durchsatz der Anwendungen im \verb|native mode|
daher nur etwa die Hälfte der Anwendungen im \verb|JVM mode|.

Im \verb|JVM mode| für statische Daten ist die reaktive Anwendung in jeder Metrik überlegen und erzielt mit 140.000 Anfragen/Sekunde
einen 102\% höheren maximalen Durchsatz.
Auch hat sie einen 31\% geringeren Speicherbedarf, eine 11\% schnellere Startzeit und kann 120\% mehr Anfragen bearbeiten,
bevor die CPU-Auslastung maximal wird.

Auch im \verb|JVM mode| für dynamische Daten mit Datenbankanbindung ist die reaktive Anwendung in jeder Metrik überlegen und erzielt
mit 36.000 Anfragen/Sekunde einen 38\% höheren maximalen Durchsatz. Darüber hinaus hat sie 24\% weniger Speicherbedarf, eine 17\% schnellere
Startzeit und kann 80\% mehr Anfragen bearbeiten, bevor die CPU-Auslastung maximal wird.

Abschließend kann zusammengefasst werden, dass die vorliegende reaktive Anwendung in jeder gemessenen Metrik bessere Werte als ihr
nicht-reaktives, blockierendes Gegenstück erzielt.
Während bei statischen Daten die Durchsatzsteigerung über 100\% beträgt, ist die
Steigerung bei dynamischen Daten mit 38\% deutlich geringer.
Allerdings ist der maximale Durchsatz bei dynamischen Daten auch
um 104.000 Anfragen/Sekunde geringer als bei statischen Daten. Die Datenbank ist bei den Lasttests der begrenzende Faktor.

Die Nachteile von reaktiven Anwendungen sind allerdings der grundlegend unterschiedliche asynchrone, eventorientierte Programmfluss,
der damit verbundene Refactoring-Aufwand sowie die schwierige Integration in bestehende Programme.

Wenn außerdem der gesamte Leistungsvorteil von reaktiven Anwendungen ausgeschöpft werden soll, ist es erforderlich
dass alle Programmschichten reaktiv sind bzw. reaktive Treiber haben, damit keine Dispatching-Kosten und Threadwechsel
verursacht werden.

\subsection{Grenzen der Arbeit}
\label{subsec:grenzen_der_arbeit}
Eine umfassende Antwort der Frage, ob reaktive Anwendungen besser skalieren als nicht-reaktive Anwendungen
mit \verb|Blocking I/O| kann nicht erfolgen, da die in dieser Arbeit vorliegenden exemplarischen \verb|workloads|, sowie
die Anwendungsarchitektur und die Datenbankkomplexität von realistischen Systemen abweichen, weswegen die gezeigten Benchmarks
wahrscheinlich nicht zu halten wären.
Darüber hinaus werden Komponenten wie Load Balancer und Cache-Server, welche die Gesamtleistung eines Systems erhöhen ,
nicht berücksichtigt.

An dieser Stelle empfehlen sich weitere Untersuchungen mit definierten, praxisnahen Testumgebungen, um die Vorteile
von reaktiven Anwendungen in der Praxis noch besser beurteilen zu können.
Da die Systemarchitekturen und Arbeitsweisen aber bei jedem Unternehmen variieren, ist eine individuelle Anpassung
für eine abschließende Beurteilung essentiell.

\subsection{Ausblick}
\label{subsec:ausblick}
Wünschenswert für die Zukunft wäre ein Vergleich der Ergebnisse mit einem ähnlichen Versuchsaufbau unter Nutzung von Project Loom,
also virtuellen Threads, die von der Laufzeitumgebung verwaltet werden und nur einen nativen Thread pro CPU-Kern nutzen.
Durch die virtuellen Threadwechsel entstehen nahezu keine Kosten. Darüber hinaus müsste der Programmablauf, wie es bei reaktiven Anwendungen
der Fall ist, nicht komplett neustrukturiert werden.

Allerdings ist ein Zeitpunkt für die Veröffentlichung von Project Loom zum derzeitigen Zeitpunkt noch nicht absehbar.
Hinzu kommt, dass eine Vielzahl an JavaEE-APIs an die Nutzung von virtuellen Threads angepasst werden müssen.
Da Project Loom allerdings direkt in die Standard-Library
integriert wird und viele Java-APIs auf virtuelle Threads umstellt, wäre der Aufwand wohl verhältnismäßg gering.

Project Loom wird eine ernsthafte Alternative zu \verb|Reactive Programming|-Bibliotheken darstellen, da sowohl der Entwicklungs-, als auch
der Integrationsaufwand in bestehende Anwendungen deutlich geringer ist.
Konzepte wie \verb|backpressure| und Nachrichten-basierte Kommunikation zur Realisierung von reaktiven Systemen
bleiben aber auch weiterhin für hochskalierbare Systeme und Komponenten interessant, die in Zukunft kein \verb|Reactive Programming| nutzen.
Dementsprechend werden auch Frameworks und Toolkits wie Vert.x für die Realisierung von reaktiven Systemkomponenten weiterhin
relevant sein.

%Glossar und Acronyme ohne Seitenzahl anzeigen, auf denen die Begriffe auftauchen
\printglossary[nonumberlist]
\printglossary[type=\acronymtype, nonumberlist]

%Literaturverzeichnis in Inhaltsverzeichnis zeigen
\addcontentsline{toc}{section}{Literaturverzeichnis}
\printbibliography
\end{document}