\newglossaryentry{benchmark}
{
    name=benchmark,
    description={Ein Benchmark im Software Performance Testing ist
            eine Metrik oder ein Bezugspunkt, mit dem Softwareprodukte oder -dienstleistungen verglichen werden können, um deren Qualität zu bewerten.
            Beliebte Benchmarks sind beispielsweise CPU-Auslastung, Speicherverbrauch, Durchsatz, Startzeiten aber auch Fehlerrate und Fehlertoleranz.
        }
}
\newglossaryentry{latenz}
{
    name=latenz,
    description={
            Die Latenz ist ein Maß dafür, wie schnell ein Server auf Anfragen des Clients reagiert.
            Die Latenz wird normalerweise in Millisekunden (ms) gemessen und wird oft als Antwortzeit bezeichnet.
            Niedrigere Zahlen bedeuten schnellere Antworten. Die Latenz wird clientseitig gemessen, von der
            Zeit, zu der die Anfrage gesendet wird, bis die Antwort eingeht. Netzwerk-Overhead ist in dieser Zahl enthalten.}
}
\newglossaryentry{durchsatz}
{
    name=durchsatz,
    description={Der Durchsatz gibt an wieviele Anfragen während eines spezifischen Zeit Intervalls von einem Server verarbeitet werden konnen.
            Der Durchsatz wird normalerweise in Anfragen pro Sekunde (requests/sec) angegeben.}
}

\newglossaryentry{perzentile}
{
    name=perzentile,
    description=
        {In der Statistik ist ein Perzentil (oder ein Zentil) ein Wert, unter den ein bestimmter Prozentsatz von Werten in seiner Häufigkeitsverteilung fällt.
            Wenn die Antwortzeit im 50. Perzentil 100ms beträgt, bedeutet dies, dass 50\% der Anfragen in 100ms oder weniger zurückgegeben wurden.
        }
}

\newglossaryentry{hdrHistogramm}{
    name=hdrHistogramm,
    description={Histogramme die das Aufnehmen und Analysieren von
            ausgewählten Daten über eine konfigurierbare, ganzzahlige Reichweite und eine konfigurierbare
            Genauigkeit innerhalb dieser Reichweite ermöglichen \parencite{HdrHistogramm}
            TODO: Add JavaEE APIS jax rs, orm, json, publisher, subscriber, IO Operationen, SPA, Servlet-Container
        },
    user1={hdrHistogramme}
}
\newacronym{http}{HTTP}{Hypertext Transfer Protocol}
\newacronym{hdrhistogram}{HdrHistogramm}{High dynamic range histogram}
\newacronym{cpu}{CPU}{Central processing unit}
\newacronym{ram}{RAM}{Random-access memory}
\newacronym{spa}{SPA}{Single page application}