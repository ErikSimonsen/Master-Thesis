\newglossaryentry{benchmark}
{
    name=benchmark,
    description={TODO}
}
\newglossaryentry{latenz}
{
    name=latenz,
    description={
            \url{https://github.com/opendocs-md/do-tutorials/blob/master/md/en/an-introduction-to-load-testing.md}
            is a measure of how fast a server responds to requests from the client.
            Typically measured in milliseconds (ms), latency is often referred to as response time.
            Lower numbers indicate faster responses. Latency is measured on the client side, from the
            time the request is sent until the response is received. Network overhead is included in this number}
}
\newglossaryentry{durchsatz}
{
    name=durchsatz,
    description={is how many requests the server can handle during a specific time interval, usually reported as requests per second}
}

\newglossaryentry{perzentile}
{
    name=perzentile,
    description={are a way of grouping results by their percentage of the whole sample set.
            If your 50th percentile response time is 100ms, that means
            50\% of the requests were returned in 100ms or less. The graph below shows why it’s useful to look at your measurements by percentile}
}
\newglossaryentry{histogram}
{
    name=,
    description={}
}
\newglossaryentry{hdrHistogramm}{
    name=hdrHistogramm,
    description={Histogramme die das Aufnehmen und Analysieren von
            ausgewählten Daten über eine konfigurierbare, ganzzahlige Reichweite und eine konfigurierbare
            Genauigkeit innerhalb dieser Reichweite ermöglichen \cite{HdrHistogramm}},
    user1={hdrHistogramme}
}
\newglossaryentry{lasttest}{
    name=lasttest,
    description={\url{https://de.wikipedia.org/wiki/Lasttest_(Computer)}}
}
\newglossaryentry{CPU}{
    name=,
    description={}
}
\newglossaryentry{RAM}{
    name=,
    description={}
}

\newacronym{http}{HTTP}{Hypertext Transfer Protocol}
\newacronym{hdrhistogram}{HdrHistogramm}{}