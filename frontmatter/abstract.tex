\begin{abstract}
    \normalsize
    \noindent
    Das Ziel der vorliegenden Masterarbeit ist es, das Konzept von reaktiver Programmierung und reaktiven Systemen,
    speziell im Java Ökosystem, zu erläutern. Zudem wird untersucht, ob eine reaktive Anwendung nach dem \verb|Reactor Pattern| das, durch
    Threadwechsel bedingte, Skalierungsproblem einer nicht-reaktiven, auf dem \verb|Thread per request|-Pattern basierenden,
    Anwendung bei hohen Lasten lösen kann.\newline

    Dafür wurden in einem Versuchsaufbau eine Reihe von Lasttests mit stetig steigenden Lasten, jeweils mit und ohne Datenbankanbindung,
    an zwei konzeptionellen Anwendungen,
    reaktiv und nicht-reaktiv, durchgeführt. Dabei wurden leistungsrelevante Metriken wie Durchsatz, Speicherbedarf und CPU-Auslastung gemessen.
    Anschließend wurden die Ergebnisse beider Anwendungen ausgewertet und miteinander verglichen.
    Beide Anwendungen wurden dabei mit dem Full-Stack, Kubernetes-nativen Java-Anwendungsframework \verb|Quarkus| implementiert.\newline

    Die Auswertung der Messergebnisse zeigt, dass die reaktive Anwendung sowohl beim Arbeiten mit statischen, als auch dynamischen Daten
    mit Datenbankanbindung in jeder gemessenen Metrik, besonders im Durchsatz, der nicht-reaktiven Anwendung überlegen ist.
    Bezüglich der Verständlichkeit, Wartbarkeit und Integrationsfähigkeit ist sie nach Auffassung des Autors gegenüber
    einer nicht-reaktiven, traditionellen Anwendung jedoch im Nachteil.
    Die Testergebnisse sind außerdem durch eine Vielzahl
    an Faktoren, wie Hardwareleistung oder Datenbankgröße, beeinflussbar, weswegen der Autor empfiehlt die Testumgebung
    an die eigenen Anforderungen anzupassen.
    Darüber hinaus werden Alternativen zum reaktiven Modell im Java-Umfeld dargestellt und bewertet.\newline

    Weiterführende Arbeiten in diesem Bereich könnten die genannten Metriken von alternativen Ansätzen, wie der Nutzung von kooperativen virtuellen Threads durch
    Project Loom, in einem ähnlichen Versuchsaufbau untersuchen und mit einer reaktiven Anwendung vergleichen.
\end{abstract}
