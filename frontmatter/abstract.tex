\begin{abstract}
    \normalsize
    \noindent
    Das Ziel der vorliegenden Masterarbeit ist es, das Konzept von reaktiver Programmmierung und reaktiven Systemen,
    speziell im Java Ökosystem, zu erläutern und zu untersuchen, ob eine reaktive Anwendung nach dem \verb|Reactor Pattern| das, durch
    Threadwechsel bedingte, Skalierungs-Problem einer nicht-reaktiven, auf dem \verb|Thread per Request|-Pattern basierenden,
    Anwendung bei hohen Lasten lösen kann.
    Dafür wurden in einem Versuchsaufbau eine Reihe von Lasttests mit stetig steigenden Lasten, sowie mit und ohne Datenbankanbindung,
    an zwei konzeptionellen Anwendungen,
    reaktiv und nicht-reaktiv, durchgeführt und Performance-bezogene Metriken wie Durchsatz, Speicherbedarf und CPU-Auslastung gemessen.
    Anschließend wurden die Ergebnisse beider Anwendungen ausgewertet und miteinander verglichen.
    Beide Anwendungen wurden dabei mit dem Full-Stack, Kubernetes-nativen Java-Anwendungsframework \verb|Quarkus| implementiert.\newline

    Die Auswertung der Messergebnisse zeigt, dass reaktive Anwendungen sowohl beim Arbeiten mit statischen, als auch dynamischen Daten mit
    Datenbankanbindung in jeder gemessenen Metrik, besonders im Durchsatz, überlegen sind, allerdings auch einige Nachteile bezüglich
    Verständlichkeit, Wartbarkeit und Integrationsfähigkeit mit sich bringen. Allerdings sind die Testergebnisse durch eine Vielzahl
    an Faktoren, wie leistungsfähigere Hardware oder Datenbankgröße, beeinflussbar, und der Autor empfiehlt die Testumgebung an die
    eigenen Anforderungen anzupassen.
    Darüber hinaus wurden Alternativen zum reaktiven Modell im Java-Umfeld dargestellt und bewertet.

    Weiterführende Arbeiten in diesem Bereich könnten die Performance von alternativen Ansätzen, wie die Nutzung von virtuellen Threads durch Project Loom,
    in einem ähnlichen Versuchsaufbau untersuchen und mit einer reaktiven Anwendung vergleichen.
\end{abstract}
