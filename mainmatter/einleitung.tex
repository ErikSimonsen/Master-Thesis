\section{Einleitung}
\label{sec:einleitung}

\subsection{Motivation}
\label{subsec:motivation}
//Problemstellung zeigen, bisheriger Ansatz durch thread switching nur begrenzt skalierbar
//Während die Kosten von Threadwechseln für, nach früheren Standards, hoch frequentierte Anwendungen lange Zeit kein Problem darstellte,
sind die Anforderungen an Webanwendungen in den letzten Jahren durch steigende Nutzerzahlen und Architekturen,
die stark auf Client-Server-Kommunikation basieren, wie Microservices und \Glspl{spag}, erheblich gestiegen.
\noindent
Sobald die Last eine realistische, kritische Grenze erreicht, erfolgen die Threadwechsel zwischen \textit{kernel threads} vom Betriebssystem
nicht mehr schnell genug, um jede Anfrage innerhalb einer akzeptablen Zeit zu bearbeiten bzw. jedem Thread ausreichend Rechenzeit zuzuweisen.
Die Kosten der Threadwechsel von \textit{kernel threads} können also zu einem Performance Problem führen, welches sich in nicht
ausreichend skalierendem \Gls{durchsatz}(*) äußert.
//Forschungsfrage: Kann Reactive Programming das Problem von modernen Workloads lösen ?
//Verwandte Arbeiten?
//Wird anhand  von \Glspl{benchmark}(*) durch Lasttests für eine 'triviale' Anwendung geprüft
//Erwähnen das alle Begriffe mit einem (*), im Glossar und im Acronym verzeichnis kurz erklärt werden, da sie möglicherweise nicht geläufig sind aber
nicht im Fließtext erläutert werden
\subsection{Aufbau}
\label{subsec:aufbau}