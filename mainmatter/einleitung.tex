\section{Einleitung}
\label{sec:einleitung}
In der folgenden Arbeit sind die ersten Vorkommen von möglicherweise nicht geläufigen Begriffen und Abkürzungen, die
im Fließtext nicht erklärt werden, mit einem \verb|(*)| gekennzeichnet.
Diese Begriffe werden im Glossar und/oder Akronymverzeichnis kurz erklärt.

\subsection{Problemstellung}
\label{subsec:problemstellung}
In den letzten Jahren sind die Anforderungen an Webanwendungen durch zunehmende Digitalisierung und entsprechend steigenden Nutzerzahlen,
sowie stark auf Client-Server Kommunikation basierenden Architekturen, wie Microservices und \Glspl{spag}(*), erheblich gestiegen.

Während geschäftskritische Anwendungen lange Zeit mit ein paar Tausend Anfragen pro Sekunde als hoch frequentiert
galten, müssen solche Webanwendungen heutzutage in der Lage sein eine vielfach höhere Last bewältigen zu können.
Darüber hinaus müssen sie \verb|skalierbar| sein, also ohne Leistungseinbußen auf variable Lasten reagieren können.

Das Standardmodell für Java-basierte Webanwendungen ist das \verb|Thread per request|-Modell.
Dabei wird jede HTTP-Anfrage an einen Kernel-Thread gebunden, welcher anschließend die Anfrage sequentiell abarbeitet.
Damit eine möglichst hohe Anzahl an Anfragen parallel abgearbeitet werden kann, wird jedem Thread nur ein Teil der verfügbaren
CPU-Rechenzeit eines CPU-Kerns zugewiesen. Sobald die Rechenzeit eines Threads abgelaufen ist, oder er durch einen blockierenden
Funktionsaufruf in einen inaktiven Zustand versetzt wird, wird der nächste Thread bearbeitet: es erfolgt ein \verb|Threadwechsel|.

Obwohl Threadwechsel, im Gegensatz zu Prozesswechseln, sehr kostengünstig sind, sind sie in diesem Modell ab einer kritischen Anzahl von HTTP-Anfragen
pro Sekunde der begrenzende Skalierungsfaktor. Die Threadwechsel zwischen den Kernel-Threads des Betriebssystems erfolgen dann nicht mehr
schnell genug um jede Anfrage bzw. jeden Thread innerhalb einer akzeptablen Zeit zu bearbeiten.
Diese Begrenzung äußert sich letztendlich darin, dass der \Gls{durchsatz}(*) der Anwendung nicht ausreichend skaliert.

\subsection{Ziel der Arbeit}
\label{subsec:ziel}
Um höhere Workloads zu bewältigen und Anwendungen skalierbarer zu machen, existieren alternative
Anwendungsmodelle, wie das in dieser Arbeit thematisierte \verb|Reactor-Pattern|.
Das Modell nutzt einen Thread pro CPU-Kern und verzichtet somit auf anwendungsbedingte Threadwechsel.
Die Voraussetzung dafür ist allerdings, dass die vorhandenen Threads niemals in einen inaktiven Zustand geraten.
Deswegen dürfen Anfragen an externe Komponenten, wie Datenbanken oder Webservices, den Thread nicht blockieren,
und der vom Ergebnis abhängige Code muss reaktiv, also erst beim tatsächlichen Eintreffen der Antwort,
ausgeführt werden: man spricht von \verb|reaktiver Programmierung|.
Reaktive Programmierung erfordert, dass die Programmlogik jeder
Anwendungsschicht asynchron und eventbasiert strukturiert wird.
Nachdem eine Anfrage an eine externe Komponente abgesetzt wurde, beginnt der Thread bereits mit der Abarbeitung der nächsten
HTTP-Anfrage. Sobald ein Resultat verfügbar ist, wird es dem Thread mithilfe eines Events mitgeteilt.
Dieser führt daraufhin den, für dieses Event hinterlegten, Code aus. Durch diese grundlegende Funktionsweise wird auf anwendungsbedingte
Threadwechsel verzichtet.

In dieser Arbeit wird untersucht, ob \verb|Reactive Programming| bzw. reaktive Anwendungen das Problem der Skalierbarkeit
für hohe Lasten praktikabel lösen können und welche alternativen Lösungsansätze es gibt.

\subsection{Vorgehensweise}
\label{subsec:vorgehensweise}
Um das Verhalten der beiden Modelle unter Last zu prüfen und miteinander zu vergleichen, werden zwei triviale Anwendungen implementiert,
reaktiv und nicht-reaktiv, und einer Reihe von Lasttests mit unterschiedlichen \verb|workloads| unterzogen.
Dabei werden verschiedene \Glsplural{benchmark}(*), wie Durchsatz, Speicherbedarf und CPU-Auslastung gemessen.
Die Anwendungen werden sowohl auf der \acrshort{jvm}(*), als auch nativ ausgeführt und jeweils mit sowie ohne Datenbankenanbindung getestet.

\subsection{Aufbau}
\label{subsec:aufbau}
Zu Beginn der Arbeit werden in Kapitel \ref{sec:grundlagen} die Grundlagen der Thematik erläutert. Diese beinhalten
Kernel- und User-Threads, blockierende und nicht-blockierende Ein-/Ausgabe-Operationen sowie die beiden darauf aufbauenden Design-Patterns
\verb|Thread per request| und \verb|Reactor|.

Anschließend wird in Kapitel \ref{section:reaktive_programmierung} die Funktionsweise von \verb|Reactive Programming|, und dessen
zugrundeliegendes Konzept von reaktiven Datenströmen erläutert. Danach werden die Vor- und Nachteile des Programmierparadigmas
genannt.
Darauffolgend werden die Eigenschaften von reaktiven Systemen und deren Beziehung untereinander beschrieben.
Zum Ende des Kapitels wird auf die Unterstützung für reaktive Anwendungen und Architekturen innerhalb des Java Ökosystem
sowie Alternativen eingegangen.

In Kapitel \ref{section:vergleich_reaktiv_blockierend} wird zuerst die Testumgebung definiert sowie Testaufbau und Testablauf beschrieben.
Im Anschluss werden die Testresultate genannt und in Form von Diagrammen dargestellt.
Abschließend erfolgt die Auswertung und Erläuterung der Testresultate.