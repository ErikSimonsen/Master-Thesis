\section{Fazit}
\label{sec:fazit}
\subsection{Zusammenfassung}
\label{subsec:zusammenfassung}
Das Ziel der Masterarbeit bestand darin, die Konzepte und Funktionsweise von reaktiver Programmierung und reaktiven Systemen, speziell im Java Ökosystem,
zu erläutern.

Dafür wurde der Leser zu Beginn mit \verb|Kernel-| und \verb|User-Threads|,
Threadwechseln, sowie \verb|Blocking I/O| und \verb|Non-blocking I/O|, welches die Grundlage
für reaktive Anwendungen bildet, vertraut gemacht.
Anschließend wurden die Eigenschaften, Funktionsweisen, Limitierungen und Anwendungszwecke der darauf aufbauenden Anwendungsmodelle
\verb|Thread per request| und \verb|Reactor| erläutert. In diesem Zuge wurden auch die grundlegenden Komponenten vom \verb|Reactor|-Pattern,
die \verb|Event Loop| und der \verb|Synchronous Event Demultiplexer|, beleuchtet, sowie das Java-Framework \verb|Netty| vorgestellt, welches
das Reactor-Pattern implementiert.
\verb|Netty| ermöglicht auf Basis des Java-Packages \verb|java.nio| die Implementierung von
hochperformanten, asynchronen Netzwerkanwendungen und bildet die Grundlage populärer Toolkits für reaktive Anwendungen.\newline

Nach der Vermittlung der Grundlagen erfolgte die Einführung des Programmierparadigmas \verb|Reactive Programming|, welches
auf dem Prinzip von \verb|Non-blocking I/O| basiert. Dabei erfolgte auch die Beschreibung der Funktionsweise und des typischen Programmablaufes
eines reaktiven Programms.
Anschließend wurden die Vor- und Nachteile von reaktiver Programmierung erläutert und zusammengefasst.

Als Nächstes wurden reaktive Datenströme und das Konzept von \verb|back-pressure| als grundlegendes Konzept von \verb|Reactive Programming|
thematisiert. Dabei wurde auch die zugrundeliegende Spezifikation \verb|Reactive Streams| erwähnt und dessen Java-Implementierung: die Flow-API.
Auf dieser Basis wurde dann das Architekturmuster eines reaktiven Systems und dessen Motivation beschrieben, sowie
das Zusammenspiel der verschiedenen Systemeigenschaften dargestellt. Zudem wurden die beiden Begriffe \verb|Reactive Programming| und Reaktives System
voneinander abgegrenzt, wobei ermittelt wurde, dass reaktive Programmierung durchaus zur Realisierung von reaktiven Systemen genutzt werden kann.
Mithilfe dieser Abgrenzung erfolgte abschließend eine Einordnung des gesamten Themenkomplexes in das Java Ökosystem.

In diesem Rahmen wurden für die Nutzung von \verb|Reactive Programming| die Bibliotheken RxJava, Project Reactor und Mutiny erwähnt, welche
alle die Java Flow-API implementieren. Für die Entwicklung von reaktiven Systemen wurden die Toolkits Vert.x, Reactor-Netty und Akka
genannt. Insbesondere Vert.x, welches auf Netty basiert, wurde anschließend näher beschrieben und dessen Erweiterung des \verb|Reactor|-Patterns
zum Multi-Reactor-Pattern thematisiert. Für die Realisierung von reaktiven Web-Applikationen wurden die beiden Frameworks Quarkus und
SpringWebFlux vorgestellt.
Sie bieten vorgefertigte Lösungen für alle Bereiche der Webentwicklung an und basieren auf einem reaktiven Unterbau durch die Nutzung
von Toolkits wie Vert.x oder Reactor-Netty. Speziell auf Quarkus und die \verb|native image|-Funktionalität des integrierbaren Graal-Compilers
wurde außerdem verstärkt eingegangen.
Abschließend wurde \verb|Project Loom| als \textit{die} Alternative zu \verb|Reactive Programming| vorgestellt und dessen Anwendungsmodell, welches
auf User-Threads mit Kooperativem Multitasking basiert, eingehend beschrieben.\newline

Darüber hinaus wurde in dieser Arbeit untersucht ob reaktive, auf \verb|Non-blocking I/O| und dem \verb|Multi-Reactor|-Modell basierende
Anwendungen das Problem der begrenzten Skalierbarkeit von nicht-reaktiven, auf \verb|Blocking I/O| und dem \verb|Thread per request|-Modell
basierenden Anwendungen durch Vermeidung von Threadwechseln lösen können, sowie alternative Anwendungsmodelle zu beschreiben.

Für diesen Zweck wurden zwei simple Anwendungen mit dem Quarkus-Framework als REST-APIs implementiert, jeweils reaktiv und nicht-reaktiv, deren
leistungsrelevante Metriken durch mehrere Reihen von Lasttests mit variierenden \verb|workloads| gemessen wurden.
Diese Metriken sind die CPU-Auslastung, der Speicherverbrauch, die Startzeit bis zur Bearbeitung der ersten Anfrage,
sowie der Durchsatz und die dazugehörige durchschnittliche Latenz.
Die Anwendungen wurden sowohl im \verb|JVM mode| als auch im \verb|native mode| jeweils mit statischen sowie dynamischen
Ressourcen, also einer Datenbankanbindung, getestet.
Anschließend wurden die Testresultate der beiden Anwendungen im jeweiligen Modus für jeden Ressourcentyp ausgewertet und
miteinander verglichen.\newline

Zusammenfassend lassen sich folgende Ergebnisse nennen:
Generell allokieren die \verb|native images| weniger Speicher und benötigen zum Starten nur wenige Millisekunden, was nur ein Bruchteil der Startzeit
von \verb|JVM|-Anwendungen darstellt. Allerdings können sie nur einen deutlich
geringeren maximalen Durchsatz aufweisen, da auf eine Vielzahl an Laufzeitoptimierungen verzichtet wird.
Sowohl für statische, als auch für dynamische Ressourcen beträgt der maximale Durchsatz der Anwendungen im \verb|native mode|
daher nur etwa die Hälfte der Anwendungen im \verb|JVM mode|.
In beiden Modi weist die reaktive Anwendung jedoch bessere Werte in allen Metriken gegenüber der nicht-reaktiven Anwendung auf.

So erreicht sie im \verb|JVM mode| für statische Daten mit 140.000 Anfragen pro Sekunde einen 102\% höheren maximalen Durchsatz.
Außerdem einen 31\% geringeren Speicherbedarf sowie eine 11\% schnellere \verb|Mean Start Time to First Request|.

Auch im \verb|JVM mode| für dynamische Daten erzielt sie mit 36.000 Anfragen/Sekunde einen 38\% höheren maximalen Durchsatz.
Darüber hinaus hat sie einen 24\% geringeren Speicherbedarf und eine 17\% schnellere \verb|Mean Start Time to First Request|.\newline
Der geringere maximale Durchsatz ist auf die Datenbank als begrenzenden Faktor zurückzuführen.

Allerdings ist die reine Performanz kein alleiniger Faktor für den Einsatz von neuen Technologien, sondern auch deren
Erlernbarkeit, Rentabilität und Integrationsfähigkeit in bestehende Lösungen.
Gerade der deutlich vom klassischen, imperativen Modell abweichende Datenfluss und die Programmstruktur, sowie die weiteren Besonderheiten
von asynchronen Programmen erschweren vielen Entwicklern das Verständnis und machen das Refactoring von bestehenden, nicht ausreichend
skalierenden Programmen und Systemen komplex und sowohl zeitlich als auch finanziell schwer einschätzbar.
Obwohl Frameworks wie Vert.x durch Mechanismen wie \verb|Dispatching| verhältnismäßig einfach in bestehende Systeme integriert werden
können, wird die bestmögliche Leistung nur erzielt, wenn alle, an der Abarbeitung einer Anfrage beteiligten, Komponenten und
Anwendungsschichten reaktiv sind und ohne \verb|Blocking I/O| auskommen. Dies ist gerade für
große Enterprise-Anwendungen ein immenser Aufwand und sollte gegen die erhofften Vorteile abgewogen werden.


\subsection{Grenzen der Arbeit}
\label{subsec:grenzen_der_arbeit}
Eine umfassende Antwort der Frage, ob reaktive Anwendungen besser skalieren als nicht-reaktive Anwendungen
mit \verb|Blocking I/O| kann nicht erfolgen, da die in dieser Arbeit verwendeten exemplarischen \verb|workloads|, sowie
die Anwendungsarchitektur und die Datenbankkomplexität von realistischen Systemen abweichen, weswegen die gezeigten Benchmarks
wahrscheinlich nicht zu halten wären.
Darüber hinaus werden Komponenten wie Load Balancer und Cache-Server, welche die Gesamtleistung eines Systems erhöhen,
nicht berücksichtigt.

An dieser Stelle empfehlen sich weitere Untersuchungen mit definierten, praxisnahen Testumgebungen, um die Vorteile
von reaktiven Anwendungen in der Praxis noch besser beurteilen zu können.
Da Systemarchitekturen und Arbeitsweisen aber bei jedem Unternehmen variieren, ist eine individuelle Anpassung
für eine abschließende Beurteilung essentiell.

\subsection{Ausblick}
\label{subsec:ausblick}
Wünschenswert für die Zukunft wäre ein Vergleich der Ergebnisse mit einem ähnlichen Versuchsaufbau unter Nutzung von Project Loom,
also virtuellen Threads mit kooperativem Multitasking, die von der Laufzeitumgebung verwaltet werden und nur einen \verb|Kernel-Thread| pro CPU-Kern nutzen.
Durch die virtuellen Threadwechsel entstehen nahezu keine Kosten. Darüber hinaus müsste der Programmablauf, wie es bei reaktiven Anwendungen
der Fall ist, nicht komplett neu strukturiert werden.

Allerdings ist ein Zeitpunkt für die Veröffentlichung von Project Loom zum derzeitigen Zeitpunkt noch nicht absehbar.
Hinzu kommt, dass womöglich eine Vielzahl an Implementierungen der Jakarta EE-APIs an die Nutzung von virtuellen Threads angepasst werden müssen.
Da Project Loom allerdings direkt in die Standard-Library
integriert wird und viele Java-APIs auf virtuelle Threads umstellt, wäre der Aufwand vermutlich verhältnismäßig gering.

Project Loom wird eine ernsthafte Alternative zu \verb|Reactive Programming|-Bibliotheken darstellen, da sowohl der Entwicklungsaufwand, als auch
der Integrationsaufwand in bestehende Anwendungen deutlich geringer ist.
Konzepte für die Entwicklung von reaktiven Systemen wie \verb|back-pressure| und nachrichtenbasierte Kommunikation bleiben aber mit hoher
Wahrscheinlichkeit auch in Zukunft für hoch skalierbare Systeme und Komponenten interessant, unabhängig davon ob sie \verb|Reactive Programming| nutzen.