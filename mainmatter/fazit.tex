\section{Fazit}
\label{sec:fazit}

\subsection{Zusammenfassung}
\label{subsec:zusammenfassung}
Ziel der Masterarbeit war es zu untersuchen, ob reaktive, auf \verb|Nonblocking I/O| und dem \verb|Multi-Reactor|-Modell basierende,
Anwendungen das Problem der begrenzten Skalierbarkeit von nicht-reaktiven, auf \verb|Blocking I/O| und dem \verb|Thread per Request|
-Modell basierenden Anwendungen durch Vermeidung von Threadwechseln lösen können, sowie mögliche Alternativen zu beschreiben.

Für diesen Zweck wurden zwei simple Anwendungen als REST-APIs geschrieben, jeweils reaktiv und nicht-reaktiv, deren
performancekritische Metriken durch mehrere Reihen von Lasttests mit variierenden \verb|workloads| gemessen wurden.
Diese Metriken sind die CPU-Auslastung, der Speicherverbrauch, die Startzeit bis zur Bearbeitung der ersten Anfrage,
sowie der Durchsatz und die durchschnittliche Latenz.
Die Anwendungen wurden sowohl im \verb|JVM mode| als auch im \verb|native mode| jeweils mit statischen Ressourcen und dynamischen
Ressourcen, also einer Datenbankanbindung, getestet.
Anschließend wurden die resultierenden Benchmarks der beiden Anwendungen im jeweiligen Modi und Ressourcentyp ausgewertet und
miteinander verglichen.
\newline
Zusammenfassend lassen sich folgende Ergebnisse nennen:\newline

Generell benötigen die \verb|native images| zum Starten nur wenige Millisekunden, was nur ein Bruchteil der Startzeit
von \verb|JVM|-Anwendungen darstellt und allokieren weniger Speicher. Allerdings können Sie nur einen deutlich
geringeren maximalen Durchsatz aufweisen, da auf eine Vielzahl an Laufzeitoptimierungen verzichtet wird.
Sowohl für statische, als auch für dynamische Daten beträgt der maximale Durchsatz der Anwendungen im \verb|native mode|
daher nur circa die Hälfte der Anwendungen im \verb|JVM mode|.

Im \verb|JVM mode| für statische Daten ist die reaktive Anwendung in jeder Metrik überlegen und erzielt mit 140.000 Anfragen/Sekunde
einen ~102\% höheren maximalen Durchsatz, sowie ~31\% weniger Speicherbedarf, eine ~11\% schnellere Startzeit und kann ~120\%
mehr Anfragen bearbeiten, bevor die CPU-Auslastung maximal wird.

Auch im \verb|JVM mode| für dynamische Daten mit Datenbankanbindung ist die reaktive Anwendung in jeder Metrik überlegen und erzielt
mit 36.000 Anfragen/Sekunde einen ~38\% höheren maximalen Durchsatz, sowie ~24\% weniger Speicherbedarf, eine ~17\% schnellere
Startzeit und kann ~80\% mehr Anfragen bearbeiten, bevor die CPU-Auslastung maximal wird.

Abschließend kann festgestellt werden, dass die vorliegende reaktive Anwendung in jeder gemessenen Metrik bessere Werte als ihr
nicht-reaktives, blockierendes Gegenstück erzielen. Während bei statischen Daten die Durchsatzsteigerung über 100\% beträgt, ist die
Steigerung bei dynamischen Daten mit 38\% deutlich geringer, allerdings ist der maximale Durchsatz bei dynamischen Daten auch
um 104.000 Anfragen/Sekunde geringer als bei statischen Daten. Die Datenbank ist bei den Lasttests der begrenzende Faktor.

Die Nachteile von reaktiven Anwendungen sind allerdings der grundlegend unterschiedliche asynchrone, eventorientierte Programmfluss
und der damit verbundene Refactoring-Aufwand, sowie die schwierige Integration in bestehende Programme.

Wenn darüber hinaus der gesamte Leistungsvorteil von reaktiven Anwendungen ausgeschöpft werden soll, ist es erforderlich
dass alle Programmschichten reaktiv sind bzw. reaktive Treiber haben, damit keine Dispatching Kosten verursacht und Threadwechsel
verursacht werden.

\subsection{grenzen der arbeit}
\label{subsec:grenzen_der_arbeit}
Eine vollständige Antwort auf die Frage, ob reaktive Anwendungen besser skalieren als nicht-reaktive Anwendungen
mit \verb|Blocking I/O| kann nicht gegeben werden, da die in dieser Arbeit vorliegenden exemplarischen \verb|workloads|, sowie
die Anwendungsarchitektur und die Datenbankkomplexität von realistischen Systemen abweichen, weswegen die gezeigten Benchmarks
wahrscheinlich nicht zu halten wären.
Darüber hinaus werden Komponenten, welche die Performance eines Gesamtsystems erhöhen wie Load Balancer und Cache-Server
nicht berücksichtigt.

An dieser Stelle empfehlen sich weitere Untersuchungen mit definierten, realitätsnäheren Testumgebungen, um die Vorteile
von reaktiven Anwendungen in der Praxis noch besser beurteilen zu können.
Da die Systemarchitekturen und Arbeitsweisen aber bei jedem Unternehmen variieren, ist für eine schlussendliche Beurteilung
eine individuelle Anpassung essentiell.

\subsection{Ausblick}
\label{subsec:ausblick}
//verweis auf project loom und andere lösungen die nicht erfordern dass anderer programmfluss und programmierparadigma
Es sind noch die folgenden Schritte notwendig…

Wünschenswert ist ein Vergleich der Ergebnisse mit …

Eine lohnende Aufgabe für die nahe Zukunft ist …
