% !TEX root = main.tex
\section {Vergleich reaktive \& imperative Anwendung}
\label{section:vergleich_reaktiv_imperativ}
//Was wird hier überhaupt verglichen, kurze Einleitung  -> Vergleich Über HTTP Benchmarks in einem bestimmten Zeitinvervall mit bestimmter
Anzahl paralleler Rrequets/sec

\subsection{Implementierung \& Systemaufbau}
\label{section:implementierung}
//TODO: Erklären wie der Code funktioniert
Erklären welche Schichten vorhanden sind, Grafiken mit den verschiedenen Layern
Quarkus- Http Layer
Vert.x/Netty (Event Loops) -> Routing Layer -> entweder Workerthread-Pool (Imperativer Test) mit IO-Thread (Reactive Test)
(Undertow (Servlet), Resteasy (Jax-Rs), Reactive Resteasy, Reactive Routes?) oder
Erstmal unabhängig von Datenbankanbindung
\url{https://quarkus.io/guides/reactive-routes}
Libraries etc. erklären, warum reactive, ORM etc. um möglich realistisch am Entwickleralltag zu bleiben (aber auch ohne möglich)
Code-Beispiele für Reactive zeigen
erwähnung von http und rest ?
Verweis auf Github Repository

\subsection{Testbedingungen}
\label{section:testbedingungen}
//TODO: Erklären welche Hardware Specs und Software Versionn nötig sind, Dinge wie SSH Zugriff

\subsection{Vorgehen / Testaufbau}
\label{section:vorgehen}
//TODO: Erklären wie das gesamte System getestet wird, welche Werkzeuge (wrk2 (warum wrk2, latenzmessung im gegensatz zu wrk), top, )
Systemaufbau (Client-Host, Server-Host, User-Host)
was genau gemessen wird (welche Metriken) und wie
diees beeeinflusst werden können
Jeden größeren Schritt erklären, Bauen der Anwendungen, Warm-Up (JIT), Workload

\subsection{Test: Statische Daten}
\label{section:statische_daten}

\subsubsection{Systemablauf}
//TODO: Grafik ähnlich zu Grafik in Implementierung aber mit Threadwechseln und exemplarisch mhrere Threads zeigen
(auch angeben wie Ergebnisse mit komplizierteren Queries aussehen könnten)
\subsubsection{Resultate}

\subsection{Test: Datenbankzugriffe}
\label{section:datenbankzugriffe}

\subsubsection{Systemablauf}
//TODO: Grafik ähnlich zu Grafik in Implementierung aber ohne Threadwechsel dafür Main-Thread mit dahinterliegender Datenbank,
das Nicht BLockieren bzw. Asynchronität verdeutlichen

\subsubsection{Resultate}

\subsection{Auswertung}
\label{section:auswertung}