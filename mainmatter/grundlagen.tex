% !TEX root = main.tex

\section{Grundlagen}
\label{section:grundlagen}

\subsection{Threads \& Prozesse in Java}
\label{sections:treads_prozesse}
In der Java-Laufzeitumgebung sind Prozesse und Threads als Betriebssystem-Prozesse realisiert, sog. \textit{native threads} oder auch \textit{Kernelthreads}.
Hierbei wird die Ausführungsreihenfolge, die Ausführungszeit und der Prozess- \& Threadwechsel
vom Scheduler \& Dispatcher des Betriebssystems übernommen \parencite{Tanenbaum2016}.
Die Threading-Abstraktion in Java bietet Entwicklern verhältnismäßig leichten Zugriff auf parallelle Programmierung und Synchronisation von Threads.\newline
Servlet-Container binden üblicherweise jede vom Webserver weitergeleitete Anfrage an einen
Thread\footnote{Aus einem, im vornherein erzeugten, Thread-Pool} im Servlet-API, welcher die jeweilige Anfrage imperativ abarbeitet
(daher auch \textit{worker thread}).\newline
Der auszuführende Code ist in diesem Ansatz an den jeweiligen Thread gekoppelt, dieser wartet bei
asynchronen Ereignissen solange, bis er eine Antwort erhält und blockiert den Thread bis dahin.

Um jede Anfrage, und somit jeden Thread, scheinbar parallel zu bearbeiten wird vom Scheduler
des Betriebssystems regelmäßig ein Kontextwechsel zwischen den Threads,
ein \textit{thread-switch}, durchgeführt. Während bei einem Prozesswechsel der gesamte Programmkontext (Adressraum, Inhalt der CPU-Register,
Seitentabelle, geöffnete Dateien und Metainformationen)
gewechselt werden muss, wird bei einem Threadwechsel lediglich der Inhalt der CPU-Register (inkl. Programmzähler) ersetzt\parencite{Brosenne2021}\parencite{Mosberger2002}.
Da der Kontextwechsel, im Fall von \textit{native threads}, durch Systemaufrufe, also vom Kernel des Betriebssystems, ausgeführt werden muss, entsteht auch
bei einem Threadwechsel ein messbarer Zeitverlust.\newline
Weitere Threadwechsel entstehen, wenn ein Thread die zugewiesene Rechenzeit nicht nutzen kann, da er noch durch ein asynchrones Ereignis
(abgesetzte Datenbankabfragen oder weitere aufgerufene Webservices) blockiert, und diese einem anderen Thread zugeteilt wird.

Während dieser Zeitverlust für hoch frequentierte Anwendungen lange Zeit kein Problem darstellte,
sind die Anforderungen an Webanwendungen in den letzten Jahren durch steigende Nutzerzahlen und Architekturen,
die stark auf Client-Server-Kommunikation basieren, erheblich gestiegen.
Ab einer gewissen Menge an Anfragen stellen die Kosten der Threadwechsel von \textit{native threads} ein Performance Bottleneck
(in Form von Durchsatz) dar.

\subsection{Reaktive Programmierung}
\label{section:reaktive_programmierung}
Reaktive Programmierung ist ein Programmierparadigma, bei dem der Programmablauf als Sequenz von asynchronen Ereignissen (Events), und
Daten als -von außen- unveränderliche (immutable) Datenströme (Streams) dargestellt werden.
Sobald es innerhalb des Datenstroms zu Änderungen kommt werden diese als Events durch einen Publisher veröffentlicht.\footnote{Änderungen in Datenströmen
    sind der quasi der Stimulus}
Die eigentliche Programmlogik wird in Funktionen ausgeführt, die auf die veröffentlichten Events hören (Subscriber), sie verarbeiten und wiederrum welche
veröffentlichen können. Die Grundidee orientiert sich am Observer-Pattern und dessen Ausprägung: dem Publish-Subscribe Pattern, erweitert diese aber
noch um die Benachrichtigungen des Subscribers:
\begin{enumerate}
    \item Sobald keine Events mehr kommen
    \item Wenn ein Fehler aufgetreten ist
\end{enumerate}
Indem Änderungen direkt propagiert werden und Subscriber keine Kontrolle über den Datenfluss haben, sondern lediglich über Änderungen informiert werden,
können Programme ohne jeglichen Zustand realisiert werden\parencite{Escoffier2017}.


Reaktive Programmierung verinnerlicht das Konzept von nicht-blockierender bzw. asynchroner Ein- und Ausgabe (I/O).
Dabei wird, statt wie bei synchroner bzw. blockierender Ein- und Ausgabe die restliche Ausführung des Programms
zu blockieren bis die Datenübertragung abgeschlossen ist, nach dem Start der Übertragung bereits begonnen die Teile des
Programms auszuführen, die nicht von dem Ergebnis der I/O Operation abhängen.

Dadurch können mehrere parallele Anfragen von dem gleichen Thread bearbeitet werden.
Methoden die blockierende I/O Operationen ausführen, wie Datenbankzugriffe oder Anfragen von externen Services,
geben beim Aufruf unverzüglich einen Publisher zurück, auf dem sich der Aufrufer registriert (subscribe).
Dadurch wird der bearbeitende Thread nicht blockiert, und kann die nächste Anfrage bearbeiten.
Sobald das Ergebnis der I/O Operation bereit ist, wird es dem Publisher in Form eines Events mitgeteilt, von diesem veröffentlicht und die Anfrage kann
vom Aufrufer bzw. Subscriber weiter bearbeitet werden.

Da durch dieses Modell der auszuführende Code nicht mehr an den jeweils ausführenden Thread gebunden wird, erlaubt es die Nutzung eines einzigen
Threads (\textit{sog. IO Thread}) statt eines \textit{Threadpools}.
Dadurch ergeben sich folgende Vorteile:

\begin{enumerate}
    \item Die Antwortenzeiten sind für eine hohe Last geringer, da deutlich weniger Threadwechsel gemacht werden müssen
    \item Der Speicherverbrauch ist geringer, da weniger Threads genutzt werden
    \item Der Grad der Paralellität ist nicht von der Anzahl der Threads begrenzt
\end{enumerate}

Allerdings gibt es auch einige gravierende Nachteile:

\begin{enumerate}
    \item Asynchroner Code ist schwieriger zu schreiben, lesen, testen und zu debuggen als imperativer Code
    \item Sehr aufwendig in bestehende klassische Anwendungen zu integrieren
    \item Reaktive Anwendungen müssen in jeder Schicht reaktiv sein (Transaktionen, Security, Datenbanktreiber)
    \item Da nur ein IO-Thread genutzt wird resultieren blockierende I/O-Operationen in der Blockierung der gesamten Anwendung
\end{enumerate}

Ein beliebtes Threading-Modell für die Verarbeitung von asynchronen Events ist die Event Loop. Sobald ein Event entsteht wird es einer Warteschlange (Queue)
in der Event Loop hinzugefügt. Solange der Main-Thread aktiv ist und die Queue noch Events enthält, wird in einer Schleife das nächste Event
abgerufen und an den, für diesen Eventtyp, registrierten Event-Handler bzw. oben beschriebenen Subscriber weitergeleitet.
Dies können beispielsweise I/O-Events sein, die signalisieren das Daten bereit zur Weiterverarbeitung sind, aber auch jegliches andere Event.
Die Event Loop wird in der Regel vom Main-Thread (in diesem Fall dem genannten IO-Thread) ausgeführt, daher darf das Verarbeiten von Events
keine blockierenden, oder zeitintensiven Operationen beinhalten\parencite{Ponge2020}.

\subsubsection{Alternativen}
In Java 1.1 wurden Threads als sog. \textit{Green Threads} implementiert. Dabei wurde die Möglichkeit Threads vom Betriebssystem verwalten zu lassen
gar nicht genutzt. Stattdessen lief die komplette JVM in einem einzigen Prozess.
\textit{Green threads} waren als \textit{user threads} implementiert \footnote{Auch als \textit{Fiber} oder \textit{virtual thread} bezeichnet},
dabei ist die Funktionalität
nicht im Kernel implementiert (wie bei \textit{kernel-/native threads}), sondern in einer Programmbibliothek im \textit{Userspace}.
Da sich das Betriebssystem nicht um das Scheduling von \textit{user threads} kümmert, wurde dies über einen eigenen Scheduling-Algorithmus der JVM
geregelt.\parencite{Oracle2010}
Ein \textit{green thread} existiert lediglich als Objekt innerhalb der JVM, und durch die virtuelle Speicherverwaltung entfallen somit
die aufwändigen Betriebssystemaufrufe beim
Erstellen eines Threads, sowie bei Thread- bzw. Kontextwechseln, denn der ausführende Main-Thread bleibt gleich.

Die Threadwechsel der \textit{user-threads} erfolgten ausschließlich innerhalb des Main-Threads, weswegen keine echte Parallelität realisiert werden konnte,
da immer nur ein Prozessorkern genutzt wurde.
Während der Vorteil dieses Modells darin lag, dass es keine 'echten' parallelen Zugriffe auf eine Resource innerhalb des JVM-Prozesses geben konnte
und die Synchronisation von Datenzugriffen daher leicht war, überwog schließlich der Umstand, dass keine Nutzung von mehreren Prozessorkernen
durch Multithreading möglich war, weswegen \textit{Green Threads} ab Java 1.3 zugunsten von \textit{native threads} entfernt wurden.

Mit dem OpenJDK Projekt \textit{Project Loom} ist die Idee von \textit{Green threads}
wieder aufgegriffen worden, allerdings nun als Ergänzung (statt Alternative) zu \textit{native threads}.
Statt alle virtuellen Threads auf dem nativen Main-Thread auszuführen, werden diese von einer geringen Anzahl an nativen \textit{worker threads},
die als Carrier eingesetzt werden, ausgeführt.
Deren Anzahl ist so gewählt, dass alle CPU Kerne durch Multithreading dauerhaft benutzt werden können
\footnote{In der Praxis laufen natürlich noch andere Prozesse auf dem Server, deren Threads auch ausgeführt werden müssen.}
, aber so wenig Kontextwechsel wie möglich ausgeführt werden müssen.
\parencite{Oracle2021} \footnote{Im Idealfall würde auf jedem CPU Kern ein \textit{worker thread} laufen,
    ohne jemals einen Thread- bzw. Kontextwechsel zu machen.}
Während ein nativer Thread in einer 64 Bit JVM ca. 1 MB für den Threadstack reserviert und zusätzlich noch Metadaten abspeichert, ist ein virtueller Thread
lediglich ein Objekt im virtuellen Speicher der JVM und benötigt sehr wenig Resourcen (da er ja im Hintergrund von einem
nativen Thread abgearbeitet wird).
Aus diesem Grund können durchaus mehrere Millionen virtueller Threads erzeugt werden (bei entsprechendem allokierten Heap-Speicher der JVM), wohingegen
die Erstellung von 10.000 nativen Threads entweder den allokierten Speicher weit überschreitet (wodurch der JVM-Prozess abstürzt) oder die Threadgrenze
des Betriebssystems überschreitet.

Sobald ein virtueller Thread nun eine blockierende I/O Operation ausführt signalisiert er dem darunterliegenden nativen Thread, dass er momentan nichts machen
kann außer Warten, und erlaubt dem nativen Thread zu einem anderen virtuellen Thread zu wechseln.

Das große Versprechen des Projektes ist außerdem, das Entwickler keine asynchronen Programmierparadigmen (wie u.A. \textit{Reactive Programming})
nutzen müssen um die beschriebenen virtuellen Threads (und die damit einhergehenden wesentlichen Performanceverbesserungen) nutzen zu können.
Um dieses Versprechen zu halten werden virtuelle Threads, statt als Bibliothek eines Drittanbieters, in Form einer eigenen JDK Version bereitgestellt.
In dieser Version wurden viele Teile der Standardbibliothek die mit I/O-Operationen arbeiten so angepasst, dass virtuelle Threads statt native Threads
genutzt werden. Auf diese Weise können I/O-Operationen, wie beispielsweise der Aufruf einer Netzwerkfunktionalität, ohne Änderungen am Programm
die virtuellen Threads nutzen und blockieren den darunterliegenden nativen Thread nicht mehr.

\subsection{Reaktive Datenströme}
\label{section.reaktive_datenströme}
In einer typischen asynchronen Verarbeitungskette von, potenziell unbegrenzten, Datenströmen
bestehend aus einem Sender und Empfänger bzw. Publisher und Subscriber kann es vorkommen,
dass der Sender Daten schneller an den Empfänger verschickt, als dieser sie verarbeiten kann.
Zwei naive Ansätze mit einer Überlastung des Empfängers umzugehen wären:
\begin{enumerate}
    \item Nur der Empfänger reagiert auf eine Überlast. Diese kann sich in einem Speicherüberlauf äußern oder, falls der Puffer des Empfängers eine Größenbeschränkung
          hat, im Verlust der empfangenen Daten
    \item Der Sender begrenzt im vornherein die Datenmenge, die er an den Empfänger schickt. Da der Sender allerdings in der Regel nicht weiß wieviel der Empfänger
          verarbeiten kann, sendet er entweder zuviel (es entsteht eine Überlast), oder er sendet zuwenig wodurch der Durchsatz geringer ist als nötig
\end{enumerate}\parencite{JavaSpektrum2015}
Die Lösung für dieses Problem wird \textit{Backpressure} genannt.
Dabei fordert der Empfänger die Daten entsprechend seiner Kapazitäten an, wodurch der Sender weiß wieviele Daten er maximal versenden darf.
Diese Mitteilung muss asynchron geschehen, da bei einer synchronen Kommunikation der Backpressure die Vorteile der asynchronen, reaktiven Datenverarbeitung
negiert würden.
Da große Anwendungen aus mehreren Schichten bestehen und somit zwischen dem Sender und Empfänger mehrere Komponenten liegen können, muss jedes
Element der Verarbeitungskette nichtblockierendes, asynchrones Verhalten implementieren, da ansonsten der Rest der Kette blockiert würde.


Aus der Intention einen Standard für die asynchrone Verarbeitung von Datenströmen mit nicht-blockierender \textit{back pressure}
zu schaffen, ging die \textit{Reactive Streams}-Initiative hervor.
Innerhalb dieser Initiative haben sich mehrere Arbeitsgruppen gebildet, welche die grundlegenden Semantiken erarbeitet haben und
sie in Form einer eigenen Java-Spezifikation namens \textit{Reactive Streams}-API implementiert und veröffentlicht haben.\parencite{ReactiveStreams}
Diese API wurde anschließend in Java 9, als Schnittstelle namens \textit{Flow-API} dem JDK hinzugefügt.

Die \textit{Flow-API} des JDK entspricht der \textit{Reactive Streams} Spezifikation und stellt (nur) Interfaces zur Verfügung mit denen eine
asynchrone, nicht blockierende Verarbeitung von (unbegrenzten) Datenströmen mit \textit{back pressure} auf der JVM implementiert werden kann.
\parencite{OracleFlow}.
\footnote{Nicht zu Verwechseln mit den Java-Streams durch die Collection-API ab Java 8. Diese sind zur Auswertungszeit in ihrer Größe begrenzt und
    nach der Abarbeitung liegt statt eines Streams eine Collection}

\subsection{Reaktive Systeme}
\label{section:reaktive_systeme}
Anforderungen an große Softwaresysteme haben sich in den letzten Jahren stark verändert:
\begin{enumerate}
    \item Antwortzeiten in Millisekunden statt im Sekundenbereich
    \item Datengrößen in Petabytes statt Gigabytes
    \item 100\% Verfügbarkeit statt stundenlange Wartungsarbeiten
    \item Deployment auf einer Vielzahl von Plattformen und cloud-basierten Clustern mit tausenden Multikernprozessoren
\end{enumerate}

Unternehmen aus verschiedenen Bereichen haben sich voneinander unabhängig an diese Kriterien angepasst und Architekturmuster
erarbeitet, mit denen robuste, belastbare und flexible Softwaresysteme entwickelt werden können, die die modernen Anforderungen
erfüllen.

2014 wurde mit dem \textit{Reactive Manifesto} versucht diese Ansätze des Systemdesigns in Form eines Manifests zusammenzuführen
und daraus allgemeingültige Systemattribute abzuleiten.
Laut dieses Manifests sind Systeme reaktiv wenn sie:
\begin{enumerate}
    \item Reaktionsschnell
    \item Widerstandsfähig (gegen Fehler)
    \item Elastisch
    \item Nachrichtengesteuert
\end{enumerate}
sind.
Solche Systeme sind, laut den Autoren, flexibler, stärker entkoppelt und würden besser skalieren als herkömmliche, nicht-reaktive Systeme.
Dies mache sie leichter zu entwickeln, zugänglicher für Veränderungen und deutlich fehlertoleranter.

Die Autoren definieren die genannten Systemeigenschaften wie folgt:

Reaktionsschnelligkeit: Das System reagiert, falls überhaupt möglich, rechtzeitig. Reaktionsgeschwindigkeit ist dabei die Grundlage von Nutzen und
Benutzbarkeit und ermöglicht das schnelle Erkennen und Behandeln von Fehlern.
Der Fokus von reaktionsschnellen Systemen liegt auf konsistenten und schnellen Antwortszeiten. Darüber hinaus schaffen sie
verlässliche Obergrenzen um eine konsistente Qualität zu erreichen.
Dieses konsistente und verlässliche Verhalten simplifiziert Fehlerbehandlung, und erhöht das Vertrauen der Benutzer.

Widerstandsfähig/Fehlertolerant: Das System bleibt auch bei Fehlern reaktionsschnell. Das gilt nicht nur geschäftskritische, hochverfügbare Systeme -
jedes System das nicht Fehlertolerant ist, wird nach Fehlern nicht mehr reaktionsfähig sein.
Widerstandsfähigkeit wird durch Redundanz, Eingrenzung, Isolation und Delegation erreicht.
Fehler werden innerhalb einer Komponente eingegrenzt und die Komponenten sind voneinander isoliert. Dadurch bleibt das Gesamtsystem stabil, selbst
wenn eine einzelne Komponente versagt.
Die Wiederherstellung jeder Komponente wird an eine andere (möglicherweise externe) Komponente deligiert, und
Hochverfügbarkeit der Komponenten wird, wo notwendig, durch Redundanz gewährleistet.

Elastisch: Das System bleibt reaktionsschnell unter variierenden Arbeitslasten. Auf Änderungen der Arbeitslast wird durch das Anpassen der
allokierten Ressourcen reagiert. Das impliziert ein Systemdesign das keine zentralen Performance-Bottlenecks oder Reibunspunkte hat, damit
Komponenten problemlos repliziert und die Last darauf verteilt werden kann.
Reaktive Systeme unterstützen prädiktive, skalierende Algorithmen zur Ressourcenberechnung,
indem Sie die Live-Messungen von Performance relevanten Systemmetriken als Eingabe nutzen.

Nachrichtengesteuert: Reaktive Systeme basieren auf dem asynchronen Austausch von Nachrichten, um die Komponenten voneinander abzugrenzen und dadurch
eine loose Kopplung, Isolation und eine transparente Lokalisierung der Komponenten zu ermöglichen.
Aufgrund dieser Abgrenzung werden Fehler als Nachrichten an andere Komponenten delegiert.
Der Ansatz jegliche Kommunikation der Komponenten durch das Übermitteln von Nachrichten zu implementieren erlaubt Elastizität,
indem er das Verteilen der Arbeitslast, und die Kontrolle der Datenströme durch das Überwachen der Nachrichtenwarteschlangen
(\textit{message queues}) und, falls nötig, Anwenden von \textit{back pressure}, erlaubt.\parencite{ReactiveSystems}

\subsection{Werkzeuge}
\subsubsection{Java Ökosystem}
Im Java Ökosystem gibt es eine Vielzahl an Frameworks, Libaries und APIs mit denen Reaktive Programmierung und reaktive Systeme umgesetzt
werden können.
Um in Java einzelne, asynchrone Prozesse zu implementieren, wird vom JDK die Future-API zur Verfügung gestellt.\parencite{OracleFuture}
Für die Verarbeitung von asynchronen (unbegrenzten) Datenströmen gibt es die Flow-API.\parencite{OracleFlow}
Da die Flow-API lediglich Interfaces bereitstellt, gibt es mehrere Implementierungen von \textit{reactive streams}.

Um reaktive Programmierung zu erleichtern, gibt es unter Anderem die folgenden Projekte:
\begin{enumerate}
    \item RxJava
    \item Spring Webflux
    \item Mutiny
\end{enumerate}
Jedes Projekt unterscheidet sich dabei in den verwendeten Klassennamen \footnote{RxJava - Observable; Mutiny - Uni, Multi; Spring Webflux - Mono, Flux},
Operatoren und dem Grad der funktionalen Programmierung.\parencite{ReactiveX, Mutiny}
Allerdings sind die meisten Frameworks und Bibliotheken interoperabel, da sie die \textit{reactive streams} Spezifikation implementieren (und damit
\textit{back pressure}), und bieten Converter-Klassen an.

Für die Entwicklung von reaktiven Systemen bieten sich mehrere Toolkits und Frameworks an.
Sie implementieren bereits Komponenten und Mechanismen wie Messaging, Event Loops, Dateizugriffe, nichtblockierende Netzwerkanwendungen, Web APIs und mehr.
Zu den populärsten gehören:
\begin{enumerate}
    \item Eclipse Vert.x
    \item Akka
    \item Project Reactor
\end{enumerate}\parencite{Vert.x, Akka, ProjectReactor}

Bei dem, in dieser Arbeit verwendeten, Framework Quarkus handelt es sich, laut Hersteller Red Hat, um ein
benutzerfreundliches, auf Entwickler abgestimmtes Java Framework, welches für Container-, Cloud- und Serverless-Umgebungen optimiert ist und nur wenig
Konfiguration benötigt.
Dabei können die Anwendungen sowohl auf einer JVM (JVM-Mode) laufen, als auch, durch native Kompilierung mit vollständigem Stack,
als \textit{native executable} (native-mode). Dafür nutzt Quarkus eine, von Oracle entwickelte, Technologie namens GraalVM. Dabei handelt es sich um
eine polyglotte, virtuelle Maschine und Laufzeitumgebung die auf dem OpenJDK basiert, und über JVMCI \textit{Java Virtual Machine Compiler Interface} den
C2-Compiler der zugrundeliegenden HotSpot-JVM durch den Graal JIT-Compiler ersetzt.\parencite{GraalVM}

Darüber hinaus verspricht Quarkus, durch seine Container-first-Philosophie, bis zu 300 Mal schnellere Startzeiten und nur ein Zehntel des
Speicherbedarfs im Vergleich zum 'traditionellen' Java, wodurch es eine signifikante Reduzierung der benötigten Ressourcen im Cloud-Umfeld bewirkt.

Des Weiteren erlaubt Quarkus die Kombination des imperativen und des reaktiven, nicht-blockierenden Programmierparadigmas.
Für die reaktive Programmierung bietet Quarkus die bereits genannte Bibliothek Mutiny.
Quarkus selber ist zudem auch reaktiv, denn 'angetrieben' wird es durch eine nicht-blockierende, reaktive Eclipse Vert.x Engine, die
alle Netzwerk I/O-Operationen verarbeitet. \parencite{QuarkusReactiveGettingStarted, Quarkus}