% !TEX root = main.tex

\section{Grundlagen}
\label{section:grundlagen}

\subsection{Threads \& Prozesse in Java}
\label{sections:treads_prozesse}
In der Java-Laufzeitumgebung sind Prozesse und Threads als Betriebssystem-Prozesse realisiert, sog. \textit{native threads} oder auch \textit{Kernelthreads}.
Hierbei wird die Ausführungsreihenfolge, die Ausführungszeit und der Prozess- \& Threadwechsel
vom Scheduler \& Dispatcher des Betriebssystems übernommen \parencite{Tanenbaum2016}.
Die Threading-Abstraktion in Java bietet Entwicklern verhältnismäßig leichten Zugriff auf parallelle Programmierung und Synchronisation von Threads.\newline
Servlet-Container binden üblicherweise jede vom Webserver weitergeleitete Anfrage an einen
Thread\footnote{Aus einem, im vornherein erzeugten, Thread-Pool} im Servlet-API, welcher die jeweilige Anfrage imperativ abarbeitet
(daher auch \textit{worker thread}).\newline
Der auszuführende Code ist in diesem Ansatz an den jeweiligen Thread gekoppelt, dieser wartet bei
asynchronen Ereignissen solange, bis er eine Antwort erhält und blockiert den Thread bis dahin.

Um jede Anfrage, und somit jeden Thread, scheinbar parallel zu bearbeiten wird vom Scheduler
des Betriebssystems regelmäßig ein Kontextwechsel zwischen den Threads,
ein \textit{thread-switch}, durchgeführt. Während bei einem Prozesswechsel der gesamte Programmkontext (Adressraum, Inhalt der CPU-Register,
Seitentabelle, geöffnete Dateien und Metainformationen)
gewechselt werden muss, wird bei einem Threadwechsel lediglich der Inhalt der CPU-Register (inkl. Programmzähler) ersetzt\parencite{Brosenne2021}.
Da der Kontextwechsel, im Fall von \textit{native threads}, durch Systemaufrufe, also vom Kernel des Betriebssystems, ausgeführt werden muss, entsteht auch
bei einem Threadwechsel ein messbarer Zeitverlust.\newline
Weitere Threadwechsel entstehen, wenn ein Thread die zugewiesene Rechenzeit nicht nutzen kann, da er noch durch ein asynchrones Ereignis
(abgesetzte Datenbankabfragen oder weitere aufgerufene Webservices) blockiert, und diese einem anderen Thread zugeteilt wird.

Während dieser Zeitverlust für hoch frequentierte Anwendungen lange Zeit kein Problem darstellte,
sind die Anforderungen an Webanwendungen in den letzten Jahren durch steigende Nutzerzahlen und Architekturen,
die stark auf Client-Server-Kommunikation basieren, erheblich gestiegen.
Ab einer gewissen Menge an Anfragen stellen die Kosten der Threadwechsel von \textit{native threads} ein Performance Bottleneck
(in Form von Durchsatz) dar.

\subsection{Reaktive Programmierung}
\label{section:reaktive_programmierung}
Reaktive Programmierung ist ein Programmierparadigma, bei dem der Programmablauf als Sequenz von asynchronen Ereignissen (Events), und
Daten als -von außen- unveränderliche (immutable) Datenströme (Streams) dargestellt werden.
Sobald es innerhalb des Datenstroms zu Änderungen kommt werden diese als Events durch einen Publisher veröffentlicht.
Die eigentliche Programmlogik wird in Funktionen ausgeführt, die auf die veröffentlichten Events hören (Subscriber), sie verarbeiten und wiederrum welche
veröffentlichen können. Die Grundidee orientiert sich am Observer-Pattern und dessen Ausprägung: dem Publish-Subscribe Pattern, erweitert diese aber
noch um die Benachrichtigungen des Subscribers:
\begin{enumerate}
    \item Sobald keine Events mehr kommen
    \item Wenn ein Fehler aufgetreten ist
\end{enumerate}

Indem Änderungen direkt propagiert werden und Subscriber keine Kontrolle über den Datenfluss haben, sondern lediglich über Änderungen informiert werden,
können Programme ohne jeglichen Zustand realisiert werden.

Reaktive Programmierung verinnerlicht das Konzept von nicht-blockierender bzw. asynchroner Ein- und Ausgabe (I/O).
Dabei wird, statt wie bei synchroner bzw. blockierender Ein- und Ausgabe die restliche Ausführung des Programms
zu blockieren bis die Datenübertragung abgeschlossen ist, nach dem Start der Übertragung bereits begonnen die Teile des
Programms auszuführen, die nicht von dem Ergebnis der I/O Operation abhängen.

Dadurch können mehrere parallele Anfragen von dem gleichen Thread bearbeitet werden.
Methoden die blockierende I/O Operationen ausführen, wie Datenbankzugriffe oder Anfragen von externen Services,
geben beim Aufruf unverzüglich einen Publisher zurück, auf dem sich der Aufrufer registriert (subscribe).
Dadurch wird der bearbeitende Thread nicht blockiert, und kann die nächste Anfrage bearbeiten.
Sobald das Ergebnis der I/O Operation bereit ist, wird es dem Publisher in Form eines Events mitgeteilt, von diesem veröffentlicht und die Anfrage kann
vom Aufrufer bzw. Subscriber weiter bearbeitet werden.

Da durch dieses Modell der auszuführende Code nicht mehr an den jeweils ausführenden Thread gebunden wird, erlaubt es die Nutzung eines einzigen
Threads (\textit{sog. IO Thread}) statt eines \textit{Threadpools}.
Dadurch ergeben sich folgende Vorteile:

\begin{enumerate}
    \item Die Antwortenzeiten sind für eine hohe Last geringer, da keine Threadwechsel gemacht werden müssen
    \item Der Speicherverbrauch ist geringer, da weniger Threads genutzt werden
    \item Der Grad der Paralellität ist nicht von der Anzahl der Threads begrenzt
\end{enumerate}

Allerdings gibt es auch einige gravierende Nachteile:

\begin{enumerate}
    \item Asynchroner Code ist schwieriger zu schreiben, lesen, testen und zu debuggen als imperativer Code
    \item Sehr aufwendig in bestehende klassische Anwendungen zu integrieren
    \item Reaktive Anwendungen müssen in jeder Schicht reaktiv sein (Transaktionen, Security, Datenbanktreiber)
    \item Da nur ein IO-Thread genutzt wird resultieren blockierende I/O-Operationen in der Blockierung der gesamten Anwendung
\end{enumerate}

Ein beliebtes Threading-Modell für die Verarbeitung von asynchronen Events ist die Event Loop. Sobald ein Event entsteht wird es einer Warteschlange (Queue)
in der Event Loop hinzugefügt. Solange der Main-Thread aktiv ist und die Queue noch Events enthält, wird in einer Schleife das nächste Event
abgerufen und an den, für diesen Eventtyp, registrierten Event-Handler bzw. oben beschriebenen Subscriber weitergeleitet.
Dies können beispielsweise I/O-Events sein, die signalisieren das Daten bereit zur Weiterverarbeitung sind, aber auch jegliches andere Event.
Die Event Loop wird in der Regel vom Main-Thread (in diesem Fall dem genannten IO-Thread) ausgeführt, daher darf das Verarbeiten von Events
keine blockierenden, oder zeitintensiven Operationen beinhalten.

\subsubsection{Alternativen}
// Project Loom (Java Threads sind keine Betriebssystem Threads mehr, also nicht mehr im Kernel Mode und das Scheduling der Threads wird von der
\footnote{Im Gegensatz zu den ursprünglich implementierten Green-Threads der JVM,
    //TODO: Verweis auf nachfolgendes Kapitel oder im Text erklären was der Unterschied zwischen
    Kernel-\& Userthreads sind und wie diese Betriebssystemsabhängig sind und unter Windows Fiber heißen}
// JVM emuliert -> kein Neuladen / Überschreiben von Registern etc. notwendig)
\subsection{Reaktive Datenströme}
\label{section.reaktive_datenströme}
\url{http://www.reactive-streams.org/}

\subsection{Reaktive Systeme}
\label{section:reaktive_systeme}
\url{https://www.reactivemanifesto.org/}

\subsubsection{Eigenschaften}

\subsubsection{Anwendungsgebiete}

\subsection{Werkzeuge}
\subsubsection{Java Ökosystem}
// CompletableFuture
// Reactive streams
// Flow Api
// JavaRx, Vert.x \& Mutiny in Quarkus
\subsubsection{Andere}

